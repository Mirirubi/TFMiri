\appendix

\chapter{Protocolo de medida\label{sec:anexo1}}
		
		
		
		\hspace{1cm}\textbf{INICIALIZACIÓN:}
	\begin{enumerate}
		\item Iniciar el programa de SensMov.
		\item Pulsar el botón \textit{\underline{INICIAR}}
	\end{enumerate}	 
		\hspace{1cm}\textbf{CALIBRACIÓN:} 
	\begin{enumerate}
		\setcounter{enumi}{2}
		\item Colocar el guante sobre una superficie plana para que le software pueda tomar la referencia global del guante cuando está en plano.
		\item Pulsar el botón \underline{\textit{MEDIR - Sobre la superficie plana}}
		\item Colocar el guante sobre la mano del paciente. Posicionar esta en posición de reposo, esta será referencia de partida que tome el software de la mano del paciente. 
		\item Pulsar el botón \underline{\textit{MEDIR - Sobre la mano del paciente}}
		\item Acompañar la mano del paciente hasta la posición de extensión que se vaya a tomar en la sesión como máximo de movimiento y mantenerla en este punto mientras se realiza el siguiente paso. 
		\item Pulsar el botón \underline{\textit{MEDIR - Extensión}}
		\item Repetir el paso nº7 con el movimiento de flexión.
		\item Pulsar el botón \underline{\textit{MEDIR - Flexión}}
		\item Dejar la mano del paciente en posición de reposo.
	\end{enumerate}
		\hspace{1cm}\textbf{MEDICIÓN DEL MOVIMIENTO:}
	\begin{enumerate}
		\setcounter{enumi}{11}
		\item Pulsar el botón \underline{\textit{MEDIR - Medición del movimiento}}
		\item Mover el interruptor a \underline{\textit{GUARDAR}} para almacenar los datos de los movimientos que se realicen a partir de la activación del interruptor. Se puede mover e interruptor entre \underline{\textit{GUARDAR}} y \underline{\textit{DESCARTAR}} siempre que se desee mientras la medición del movimiento este activa.  	
		\item Realizar los ejercicio de rehabilitación con normalidad
		\item Pulsar el botón \underline{\textit{PAUSAR}}
	\end{enumerate}
		\hspace{1cm}\textbf{FINALIZACIÓN:}
	\begin{enumerate}
		\setcounter{enumi}{16}
		\item Pulsar el botón \underline{\textit{ARCHIVAR RESULTADOS}}
		\item Pulsar el botón \underline{\textit{SALIR}}
		\item Elegir en la pestaña emergente el nombre de archivo y carpeta para el informe de la sesión.
	\end{enumerate}


