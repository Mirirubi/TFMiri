\chapter{Solución con sensores de fibra FBG\label{sec:FBG}}
%------------------------------
%------___SOLUCIÓN_FBG___------
%------------------------------

\label{sec:FBG3}

 Como primer prototipo se ha estudiado y llevado a cabo un guante cuyo funcionamiento se basa en los sensores de fibra FBG. 
 
 El prototipo consiste en una sección de PDMS con forma de huella de mano que tiene embebida una red en fibra de Bragg. Para la obtención, procesado y visualización de los resultados medidos se emplea el entorno de desarrollo LabVIEW.



%--Marco conceptual
\section{Marco conceptual}
\label{sec:marco3}

Este apartado tiene por finalidad realizar una clara exposición de los conceptos teóricos fundamentales para la comprensión del diseño llevado a cabo. 


%--FIBRA ÓPTICA
\subsection{Fibra óptica}
\label{sec:fibra3}

	%-- ¿Qué es la fibra óptica y la comunicación óptica?
	La fibra óptica es una hebra de material dieléctrico, así cómo el vidrio (sílice) o el polímero acrílico. 
	Se emplea como medio de propagación de señales luminosas. Es decir, para transmitir ondas electromagnéticas del espectro óptico: regiones espectrales de infrarrojo, luz visible y ultravioleta. En la siguiente imagen (figura \ref{fig:espectroOptico}) se puede observar dentro del espectro electromagnético dónde se sitúa el espectro óptico.	 
	
	\begin{figure}[H]
		\centering
		\includegraphics[width=0.95\textwidth]{./img/espectrooptico}
		\caption{Espectro electromagnético en frecuencia.}
		\label{fig:espectroOptico}
	\end{figure}
	
	%--Ventanas de comunicación por FO
	Cabe destacar que dentro del espectro óptico las longitudes de onda habituales para comunicación en fibra óptica están entre los 700nm y 1600nm. Estas se dividen en rangos con mejores características para la transmisión, denominadas ventanas de comunicación. Como se muestra en la figura \ref{fig:ventanaOptica}, son tres las ventanas más utilizadas,\cite{ventanasFO}:
 	\begin{table}[H]
		%\centering
		\hspace{2cm}
		\renewcommand{\arraystretch}{2}
		\begin{tabular}{rrl}
			\textbf{1ª ventana}& 800 a  900 nm  & $\longmapsto$ $\,$ longitud de onda utilizada = 850nm  \\
			\textbf{2ª ventana}& 1250 a 1350 nm & $\longmapsto$ $\,$ longitud de onda utilizada = 1310nm  \\
			\textbf{3ª ventana}& 1500 a 1600 nm & $\longmapsto$ $\,$ longitud de onda utilizada = 1550nm   \\ 
		\end{tabular} 
	\end{table}

	 \begin{figure}[H]
	 	\centering
	 	\includegraphics[width=0.5\textwidth]{./img/ventana}
	 	\caption{Longitud de onda fibra óptica junto con el espectro visible. \cite{ventanasFO}}
	 	\label{fig:ventanaOptica}
	 \end{figure}
 	La razón de que las ventanas de comunicación utilizadas se sitúen en las frecuencias indicadas reside en los diferentes comportamientos que tiene la atenuación de las señales en función de la longitud de onda (ver figura \ref{fig:perdidasFrec}). Existen algunas zonas dónde la atenuación es mínima, coincidiendo con la segunda y la tercera ventana. En cambio, en la zona correspondiente a la primera ventana las pérdidas no son mínimas, pero sí que se mantienen constantes. 	
 	
 	\begin{figure}[H]
 		\centering
 		\includegraphics[width=0.8\textwidth]{./img/perdidasFrec}
 		\caption{Atenuación(dB/Km) frente a longitud de onda  $\lambda$ (nm) \cite{imgRadioModo}}
 		\label{fig:perdidasFrec}
 	\end{figure}

 %-- Características físicas de la fibra
 En cuanto a las propiedades físicas de la fibra óptica, son bastante delicadas ya que su grosor no supera por mucho al diámetro del cabello humano y se obtiene de la extrusión del sílice, SiO\textsubscript{2} , es decir, se trata de un filamento de vidrio muy fino. Es por ello que es la fibra óptica estándar está rodeada de una cubierta protectora. 
 
  \begin{figure}[H]
  	\centering
  	\includegraphics[width=0.87\textwidth]{./img/capas-fibra2}
  	\caption{Capas fibra óptica \cite{imgNucleoFibra,imgCapasFO}} 
  	\label{fig:capasFibra}
  \end{figure} 
  
  
  
 La fibra óptica estándar cuenta varias capas (figura \ref{fig:capasFibra}): núcleo, revestimiento y cubierta (o buffer).  Si la aplicación lo permite, conviene proteger la fibra con más capas externas. En la imagen anterior la fibra está además protegida por un material de refuerzo (fibra de aramido) y una envoltura (PVC).
 
 Tanto el núcleo cómo el revestimiento forman el medio por el cual se propaga la luz. Estas dos capas son tan finas que forman un filamento flexible, pero muy delicado, puesto que es muy propenso a romperse ante dobleces u otras manipulaciones externas. Por ello el resto de las capas son tambien importantes por proporcionar a la fibra protección y haciendo posible su utilización es escenarios de despliegue.
 
 %-- Fabricación FO
 La fabricación de la fibra óptica es un proceso de alta tecnología. Es importante mantener la pureza y la regularidad del núcleo. Esto es complejo, puesto que estamos hablando en algunos casos de núcleos de un grosor entorno a las 8 micras (en fibras monomodo). El grosor estándar de la fibra es de 125 micras(una micra equivale a una millonésima parte de un metro). Para conseguir este resultado el proceso de fabricación consiste en reproducir a escala macroscópica la estructura de la fibra que se quiere obtener. Esta reproducción a gran escala de la fibra deseada se le denomina preforma. Una vez se tiene la preforma, esta se va fundiendo y estirando hasta alcanzar el filamento del diámetro deseado. De una preforma se pueden sacar kilómetros de fibra. Para fabricar la preforma se parte de una barra de vidrio hueca (el vidrio que formará el recubrimiento) y se baña en un gas que contiene unas partículas (lo que formará el núcleo). Al calentar a mil grados, las partículas comienzan a fundirse hasta que el tubo colapsa y forma una vara maciza, que es la preforma. Para fundirla y estirarla esta se coloca verticalmente y se calienta. La complejidad de esta fase reside en mantener constante el flujo y el diámetro del hilo resultante. Además durante esta fase se aprovecha para crear una capa protectora sobre el vidrio (cubierta en la figura \ref{fig:capasFibra}). Finalmente los kilómetros de fibra óptica se enrollan en grandes bobinas. \cite{fabricacionFO}
 
 %-- Fibra Monomodo y multimodo
  \begin{figure}[H]
 	\centering
 	\includegraphics[width=0.6\textwidth]{./img/MM-SM}
 	\caption{Relación grosor fibra multimodo (MM) y monomodo (SM) \cite{imgRadioModo} } 
 	\label{fig:modoMonoMulti}
 \end{figure} 
 
  Dependiendo de la relación de diámetro entre el núcleo y el revestimiento, la fibra será monomodo o multimodo (figura \ref{fig:modoMonoMulti}). Esta diferencia afecta a la propagación de la luz dentro de la guía de onda (figuras \ref{fig:guiaMM}, \ref{fig:guiaSM}, \ref{fig:indiceMultimodo}). Ya se ha comentado que el diámetro de la fibra es de aproximadamente 125 micras. En el caso de las fibras monomodo, el núcleo de estas tiene un diámetro tan pequeño (en torno a 8 micras) que la luz solo puede propagarse en un sólo modo (rayo). Sin embargo, en el caso de las fibras multimodo, al poseer un núcleo mayor (entre 50 o 62.5 micras) soportan la transmisión el múltiples modos, es decir, los rayos de luz viajan en muchas direcciones a través de este. \cite{FOA} 
   
   	\begin{figure}[H]
		\centering
		\includegraphics[width=0.7\textwidth]{./img/guiaMM}
		\caption{Corte transversal fibra multimodo en transmisión de luz. \cite{imgMonoMulti} } 
		\label{fig:guiaMM}
	\end{figure} 
  	\begin{figure}[H]
		\centering
		\includegraphics[width=0.7\textwidth]{./img/guiaSM}
		\caption{Corte transversal fibra monomodo en transmisión de luz. \cite{imgMonoMulti} } 
		\label{fig:guiaSM}
	\end{figure}  
 	
 	%-- Relación monomodo y multimodo con las longitudes de onda
 	Relacionando los tipos de fibras ópticas con las ventanas en las que trabajan, las fibras multimodo suelen trabajar en primera y segunda ventana, mientras que las fibras monomodo en segunda y tercera. 

	  	\begin{figure}[H]
		\centering
		\includegraphics[width=0.5\textwidth]{./img/transFO-esc-grad}
		\caption{Disposición rayos. Multimodo (indice escalonado y gradual) y monomodo. \cite{FOA} } 
		\label{fig:indiceMultimodo}
		\end{figure}
	
 	%-- Otros tipos de FO
 	Además existen otros tipos de fibras menos comunes: la fibra de plástico (POF) y la fibra de sílice con revestimiento de plástico (HCS/PCS). La primera tiene un núcleo de gran diámetro ($1mm$ aproximadamente), puede utilizarse para redes de distancia corta y de baja velocidad. Las fibras de sílice con revestimiento de plástico tienen un núcleo más pequeño ($200\mu m$ aproximadamente) que las fibras de plástico. Estos dos últimos tipos de fibra multimodo generalmente son de índice escalonado, mientras que el resto de fibras multimodo suelen ser de índice gradual. En la figura \ref{fig:indiceMultimodo} se observa como es la diferencia en la distribución de los rayos en un caso y en el otro. En cuanto al tamaño de las fibras, la figura \ref{fig:otrosTiposFO} se representan las diferentes relaciones de tamaños entre los cinco tipos de fibra vistos. \cite{FOA}
 	 	
 	 \begin{figure}[H]
 	 	\centering
 	 	\includegraphics[width=0.7\textwidth]{./img/tiposFO}
 	 	\caption{Relación tamaños fibras ópticas. \cite{FOA} } 
 	 	\label{fig:otrosTiposFO}
 	 \end{figure} 	
  	
 %-- PROPAGACIÓN DE LA LUZ	
 %-- Diferenci de indices de refracción
 La diferencia de índices de refracción entre las capas centrales de la fibra son las que permiten la propagación de la luz a través de esta. El núcleo tiene un mayor indice de refracción que el revestimiento, lo que genera que los rayos de luz se curven a medida que pasan del núcleo al revestimiento, generando una ``reflexión interna total'' en la fibra.% La siguiente imagen (figura \ref{fig:TIR}) sirve para explicar claramente este concepto:
 
 \begin{figure}[H]
 	\centering
 	\includegraphics[width=0.66\textwidth]{./img/TIR}
 	\caption{Ángulo crítico y reflexión interna total. \cite{geometriaBasicaFP} } 
 	\label{fig:TIR}
 \end{figure}
 
 %Reflexión interna total
 En la figura \ref{fig:TIR} se ilustran cuatro rayos que se originan en el puno 0, lo que sería el núcleo de la fibra óptica (dónde el indice de refracción \textit{n\textsubscript{i}} es mayor). 
 %Entre los cuatro rayos varía el ángulo con el que estos inciden sobre el revestimiento. En el caso del rayo 1 la incidencia es normal, no existe reflexión, mientras que en el rayo 2 sí que existe reflexión. Sin embargo, en el caso del rayo 3 se tiene el ángulo crítico \textit{i\textsubscript{c}} que es lo suficientemente grande como para que el rayo reflejado se propague a lo largo de la interfaz entre los dos medios, quedando atrapado.
 El rayo número 4 incide con un ángulo \textit{i} superior al ángulo crítico (ecuación \ref{eq:angCritico}), \textit{i\textsubscript{c}}, consiguiendo que se refleje totalmente en el mismo medio del que incide. Este rayo obedece a la ley de reflexión, siendo su ángulo de reflexión exactamente igual a su ángulo de incidencia. La figura representa el denominado efecto de \textit{``Reflexión interna total''}, necesario para que suceda la transmisión de señales lumínicas en la fibra óptica. 

	\begin{equation}
		\label{eq:angCritico}
		\hat{i}\textsubscript{c} =  \sin^{-1}\left(\dfrac{n\textsubscript{r}}{n\textsubscript{i}}\right)
	\end{equation}

 La reflexión interna total atrapa la luz hasta cierto ángulo en el núcleo, definiendo la apertura numérica a la que hay que asegurarse de penetrar la luz para que se de el fenómeno de reflexión interna total. Así se fuerza a que la mayoría de los rayos de luz incidan sobre la interfaz y se reflejen, permitiendo la transmisión de la señal lumínica. 
 
 %-- TRANSMISIÓN EN LA FIBRA ÓPTICA	
 \subsection{Transmisión de señales a través de la fibra óptica} %Tipos de sensores ópticos
 \label{sec:transmision3}
 
 %Sistemas de propagación
   \begin{figure}[H]
 	\centering
 	\includegraphics[width=1\textwidth]{./img/TxFOp2p}
 	\caption{Transmisión punto a punto de señales a través de fibra óptica. \cite{txFO} } 
 	\label{fig:TxFOp2p}
 \end{figure} 
 
 Los sistemas de propagación de señales luminosas a través de la fibra óptica componen un medio de transmisión de datos rápido y fiable. En la figura \ref{fig:TxFOp2p} se plasma el procedimiento que sigue la transmisión de datos en un sistema óptico y los elementos que lo componen. Previa a la propagación a través de un medio óptico de una señal eléctrica (analógica o digital) es necesario realizar una conversión de esta a señal óptica. Esto genera una señal óptica a partir de una señal eléctrica en el emisor o fuente de luz situado en el extremo inicial de la comunicación. Realizada la conversión, la señal es transmitida a lo largo de la fibra óptica. Según las características del escenario puede haber una o varias uniones entre fibras a lo largo del canal. Estás pueden realizarse empalmando o utilizando conectores. Una vez la señal óptica atraviesa todo el canal, llega al detector, dónde sucede el proceso inverso al ocurrido en el emisor y a la salida del sistema completo se tiene la señal eléctrica. Esta corresponde a la señal introducida al sistema con una pequeña posibilidad de haber sufrido pérdidas o atenuación debido a la impureza de la fibra, la distancia, las conexiones entre elementos del sistema, etc. Estas modificaciones de la señal de entrada pueden ser contrarrestadas o solventadas en recepción sin suponer un impedimento a una comunicación exitosa.    
  
 Veamos por separado los elementos dibujados en la figura \ref{fig:TxFOp2p}:
 	\begin{itemize}
 	%-Emisores (Transmisión)
 		\item \textit{\textbf{Emisores (Transmisión)}}	
 		% \hspace{0.2cm} 
 		Los emisores de luz forman un papel imprescindible en la transmisión de señales a través de la fibra óptica. Se encargan de convertir la señal eléctrica a señal luminosa, para que esta se pueda propagar por el canal óptico según lo esperado. Principalmente hay dos tipos de fuentes de luz: diodos LED o diodos láser. Dentro de los de tipo láser se distinguen otros tres tipos: láser fabry-perot (\textit{FP}), láser de retroalimentación distribuida (\textit{(DFB)}) y láser de cavidad vertical y emisión superficial (\textit{VCSEL}). Unas de las condiciones más importantes que deben cumplir las fuentes de luz son que operen en la longitud de onda adecuada, se puedan de modular lo suficientemente rápido para transmitir datos y poder acoplarse de forma eficiente a la fibra. \cite{TransRecepFO}
 		
 		 \begin{figure}[H]
 			\centering
 			\includegraphics[width=0.35\textwidth]{./img/led-laser}
 			\caption{Relación entre potencia y longitud de onda diodo LED frente a diodo láser. \cite{FOA} } 
 			\label{fig:ledVsLaser}
 		\end{figure} 
 	 	\begin{figure}[H]
 			\centering
 			\includegraphics[width=0.55\textwidth]{./img/led-laserEMISION}
 			\caption{Corte transversal de fibra con emisión desde diodo LED frente a diodo láser. \cite{FOA} } 
 			\label{fig:corteledVsLaser}
 		\end{figure} 
 		
 		\begin{itemize}
 			\item \textbf{LED} - \textit{Light Emitting Diodes}\\
 			Consiste en una fuente de luz incoherente. Es una fuente de luz de mayor ancho de banda de operación que en el caso de los diodos láser (véase figura \ref{fig:ledVsLaser} y \ref{fig:corteledVsLaser}). Tiene un espectro de emisión entre los 30-100 nm. En función del material semiconductor con el que se fabrique se pueden emitir desde luz ultravioleta hasta infrarrojos.
 			%Para conseguir emitir en un espectro reducido, es necesario polarizarlo y hacerle circular corriente eléctrica. Pueden ser modulados sin dificultad hasta velocidades de 100-200 Mb/s y en algunos casos hasta velocidades de 1 Gb/s. 
 			
 			%Además, tienen un comportamiento simple, de fácil fabricación y tienen un coste bajo en comparación a otras fuentes. Su circuitería de alimentación y control es muy sencilla, debido a los bajos niveles de corriente que son necesarios para que funcione el dispositivo y a su relativa inmunidad frente a variaciones de la temperatura. Necesitan baja potencia de alimentación. Su geometría y patrón de radiación es de alta divergencia, el acoplo de luz a la fibra óptica monomodo es difícil, especialmente en los LED de emisión superficial. Son dispositivos fiables, ya que no sufren la degradación de tipo catastrófico y son menos sensibles que los diodos láser a la degradación por envejecimiento. \\
 			

 			
 			\item \textbf{Láser} - \textit{Light Amplification by Stimulated Emission of Radiation}\\
 			Consiste en un tipo de fuente de luz coherente. Tienen mayor capacidad de transmisión de luz, concentran la luz en forma de haces estrechos pero potentes, es decir, emiten de manera direccional e intensa. %Es por ello que se suele advertir que mirar con los ojos directamente a este tipo de emisiones es peligroso para la vista.
 			(Véanse figuras \ref{fig:ledVsLaser} y \ref{fig:corteledVsLaser}).
 			% Pueden ser modulados en frecuencia y a gran velocidad. Su geometría y patrón de radiación es de relativamente baja divergencia, el acoplo de luz a la fibra óptica monomodo es bastante eficiente. Su composición es más compleja que la del diodo LED. Para conseguir dicha directividad y potencia, internamente la fuente tiene cavidades que combinan medio activo y espejos. Por ejemplo, necesita tener un circuito de realimentación para su control, debido a que se ve afectado por las variaciones de temperaturay a las reflexiones que pueda provocar la potencia óptica incidente en su salida. Debido a ello, son complejos y costosos de fabricar. 
 			
 				%	\textcolor{rositaoscuro}{//- Resumen Los láseres se basan en emisión estimulada, conseguida formando cavidades que combinan medio activo y espejos para la realimentación. Son de mayores prestaciones que los LED (mayor coherencia, más directivos, mayor potencia de salida,...), pero más complejos y caros. Se pueden usar en fibras monomodo. 
 				
 			Los hay de dos tipos:	
 			\begin{itemize}
 				\item Láseres multimodo (Fabry-Perot) 
 				\item Láseres monomodo (DFB, DBR)
 			\end{itemize}
 			
 		\end{itemize}
  		
  		La tabla \ref{tabla:caractFuentes} compara las características de las diferentes fuentes mencionadas, permitiendo ver también las diferencias entre los diferentes tipos de diodos láser citados.
  		
  		
  		\begin{table}[H]
  			\begin{tabular}{l|c|c|c|c|}
  				\cline{2-5}
  				& \cellcolor[HTML]{EFEFEF}\textbf{LED} & \cellcolor[HTML]{EFEFEF}\textbf{LD F-P} & \cellcolor[HTML]{EFEFEF}\textbf{LD\_DFB} & \cellcolor[HTML]{EFEFEF}\textbf{VCSEL} \\ \hline
  				\multicolumn{1}{|l|}{\cellcolor[HTML]{EFEFEF}Espectro emisión}                 & Ancho                                & Medio                                   & \multicolumn{2}{c|}{Estrecho}                                                     \\ \hline
  				\multicolumn{1}{|l|}{\cellcolor[HTML]{EFEFEF}Directividad}                     & Muy divergente                       & \multicolumn{3}{c|}{Directivo}                                                                                              \\ \hline
  				\multicolumn{1}{|l|}{\cellcolor[HTML]{EFEFEF}Potencia}                         & Baja                                 & \multicolumn{3}{c|}{Alta}                                                                                                   \\ \hline
  				\multicolumn{1}{|l|}{\cellcolor[HTML]{EFEFEF}Velocidad / BW modulación}        & Varios cientos de MHz                & \multicolumn{3}{c|}{Varias decenas de GHz}                                                                                  \\ \hline
  				\multicolumn{1}{|l|}{\cellcolor[HTML]{EFEFEF}Acoplo a la fibra}                & MMF                                  & \multicolumn{2}{c|}{SMF}                                                           & MMF                                    \\ \hline
  				\multicolumn{1}{|l|}{\cellcolor[HTML]{EFEFEF}Curva I-P}                        & Sin $I_{umbral}$; baja pendiente          & \multicolumn{3}{c|}{Con $I_{umbral}$; alta pendiente}                                                                            \\ \hline
  				\multicolumn{1}{|l|}{\cellcolor[HTML]{EFEFEF}Dependencia con temperatura}      & Baja                                 & \multicolumn{3}{c|}{Alta}                                                                                                   \\ \hline
  				\multicolumn{1}{|l|}{\cellcolor[HTML]{EFEFEF}Circuitos electrónicos asociados} & Sencillos                            & \multicolumn{3}{c|}{Complejos}                                                                                              \\ \hline
  				\multicolumn{1}{|l|}{\cellcolor[HTML]{EFEFEF}Seguridad para la vista}          & No peligroso                         & \multicolumn{3}{c|}{Potencialmente dañino}                                                                                  \\ \hline
  				\multicolumn{1}{|l|}{\cellcolor[HTML]{EFEFEF}Tiempo vida útil}                 & Alto                                 & \multicolumn{3}{c|}{Medio (suficiente)}                                                                                     \\ \hline
  				\multicolumn{1}{|l|}{\cellcolor[HTML]{EFEFEF}Coste}                            & Bajo                                 & Medio                                   & Alto                                     & Bajo                                   \\ \hline
  				\multicolumn{1}{|l|}{\cellcolor[HTML]{EFEFEF}Ventana operación}                & 1ª, 2ª                               & \multicolumn{2}{c|}{2ª, 3ª}                                                        & 1ª, 2ª                                 \\ \hline
  			\end{tabular}
  		\caption{Tabla características fuentes de ópticas}
  		\label{tabla:caractFuentes}
  		\end{table}
  	
   	%-Detectores (Recepción)		
 		\item \textit{\textbf{Detectores (Recepción)}}
 			
 	%En cuanto al proceso de recepción, consiste en que la señal llega al final del canal y se tiene que dar el proceso inverso que en transmisión. 
 	Los detectores tienen la función de convertir las señales ópticas a señales eléctricas para recuperar la información. Son diodos semiconductores encargados de polarizar inversamente la polarización realizada en el diodo emisor.  Al igual que pasaba en la transmisión, existe detección coherente o incoherente. 
 	
 	El componente del receptor que realiza la conversión óptico-eléctrica es el fotodetector. Se distinguen varios tipos, los más comunes son: los fotodiodos PIN y los fotodiodos de efecto de avalancha (\textit{APD}).
 	La figura \ref{fig:diagRX} representa un diagrama de bloques típico de un receptor. %Aunque merece la pena mencionar los fotoespectrómetros, formados por células CCD integradas y una estructurá optica tan precisa que es capaz de detectar la potencia que la luz recibida proporciona en cada longitud de onda. Volviendo a los receptores PIN y APD, además del fotodetector, el detector puede estar compuesto por un pre-amplificador óptico, pero casi siempre tendrá un pre-amplificador eléctrico para amplificar la señal eléctrica obtenida del fotodiodo, de muy baja corriente. El receptor lo podrá formar también (figura \ref{fig:diagRX}) un filtro óptico, para tomar solo en cuenta las frecuencias de interés. Posterior a la pre-amplificación eléctrica, los elementos de procesado de la señal que se incluyan dependerán del si las señales empleadas por el sistema eléctrico son analógicas o digitales.
 		
	 	\begin{figure}[H]
	 		\centering
	 		\includegraphics[width=0.65\textwidth]{./img/diagRX}
	 		\caption{Diagrama de bloque receptor} 
	 		\label{fig:diagRX}
	 	\end{figure} 
 	
 		%Idealmente los fotodetectores deberián ser altamente sensibles, con alta velocidad de respuesta, poco ruidosos, compactos, robustos y de respuesta lineal. En la práctica es complicado conseguir todos estos atributos en conjunto. Por ejemplo, los APD requieren menor potencia óptica para funcionar que los diodos PIN, pero cuatro veces mayor voltaje de alimentación.
 		
 	
 	%-Conectores y empalmes
 		\item \textit{\textbf{Conectores y empalmes}}\\

 		%Cabe hablar también de los conectores ópticos y empalmes utilizados para unir fibra óptica a fibra óptica u otros elementos del sistema. Teniendo una fibra óptica terminada en algún tipo de conector esta se puede unir a elementos como emisores, receptores, multiplexores y otros elementos, incluso se puede unir a otra fibra terminada en conector con un adaptador macho a macho. Además, para unir dos trozos de fibra es posible realizar una fusión que empalme ambas terminaciones de la fibra. 
 		La principal diferencia entre estos dos tipos de uniones es que los conectores unen de manera no permanente, mientras que los empalmes son uniones permanentes. 
 		
 		\begin{figure}[H]
 			\centering
 			\includegraphics[width=0.95\textwidth]{./img/tiposConect2}
 			\caption{Tipos de conectores de fibra óptica. \cite{TipConectoresFO}} 
 			\label{fig:tiposConect}
 		\end{figure} 
 		
 		Es importante que las uniones no afecten a la calidad de la transmisión, es decir, deben garantizar bajas pérdidas de conectividad. En la figura \ref{fig:tiposConect} se muestra algunos de los conectores más empleados, cada uno de ellos suelen utilizarse en diferentes aplicaciones según sus características. %Por ejemplo los utilizados en este trabajo, los conectores FC (\textit{Ferule Connector}) se suelen utiliza tanto en montaje de laboratorio como de campo y sirven en fibras monomodo y multimodo. Además tienen unas pérdidas de inserción IL < 0.34 dB en fibras multimodo y IL < 0.15 dB en fibras monomodo. \cite{TipConectoresFO}\\
 		\begin{figure}[H]
	 		\centering
	 		\includegraphics[width=0.85\textwidth]{./img/perdidasOpticas}
	 		\caption{Causas de pérdida óptica en las uniones. \cite{FOAconect}} 
	 		\label{fig:conectLoss}
		\end{figure}
 		%Las pérdidas se pueden reducir cuando los núcleos de las dos fibras son idénticos, están alineados de manera perfecta y se tocan entre sí.%, los conectores y los empalmes se realizaron adecuadamente y no hay suciedad en la unión. 
 		En la figura \ref{fig:conectLoss} se pueden observar diferentes causas de pérdidas en las conexiones de fibra óptica. Además, los conectores pueden tener diferentes formas de terminaciones, lo que afecta también a las pérdidas resultantes (figura \ref{fig:tiposPulidos}).
 		
 		El extremo de la fibra tiene que estar debidamente limpio y pulido para reducir al máximo las pérdidas, ya que una superficie aspera puede dispersar o absorber luz.		
   		\begin{figure}[H]
   			\centering
   			\includegraphics[width=0.55\textwidth]{./img/conectoresPulido}
   			\caption{Tipos de conectores según su pulido. \cite{TipConectoresFO}} 
   			\label{fig:tiposPulidos}
   		\end{figure}

 		Los empalmes crean una unión permanente entre dos fibras. Hay dos tipos: por fusión y mecánicos. En la figura \ref{fig:empalmeTipo} se observa en primer lugar un empalme por fusión y los demás son mecánicos. Los más realizados son los empalmes por fusión por la fiabilidad y la robustez de la unión, así como por brindar pérdidas y reflectancias menores.
 		
 		\begin{figure}[H]
 			\centering
 			\includegraphics[width=0.5\textwidth]{./img/tiposEmpalme}
 			\caption{Tipos de empalmes. \cite{FOAconect}} 
 			\label{fig:empalmeTipo}
 		\end{figure} 
 	
		
 

 	 \end{itemize}					

%-- SENSORES ÓPTICOS	
\subsection{Sensores ópticos} %Tipos de sensores ópticos
\label{sec:sensores3}

		
		Existen en el mercado diversos tipos de sensores ópticos. Estos se pueden clasificar atendiendo a diversos aspectos. A continuación se exponen dos clasificaciones básicas\cite{sensoresOpticos}: 

			\begin{itemize}
				\item[$\cdot$] \underline{En función de la naturaleza del parámetro a cuantificar}
				\begin{itemize}
					\item Sensores químicos: Sirven para detectar variación de cantidad de ciertos componentes químicos. Además en este grupo se incluyen los biosensores.
					\item Sensores físicos: Utilizados para medir parámetros físicos (temperatura, presión, espesor, etc.)
				\end{itemize}
				\item[$\cdot$] \underline{En función de la naturaleza de la propiedad óptica medida}
				\begin{itemize}
					\item Sensores de absorbencia
					\item Sensores de reflectancia
					\item Sensores se luminiscencia (fluorescencia, quimioluminiscencia y bioluminiscencia)
					\item Sensores de dispersión Raman
					\item Sensores de índice de refracción
					\item etc.
				\end{itemize}
				\end{itemize}
			
			
				En este trabajo se van a utilizar sensores de difracción de Bragg, que son sensores que cuantifican parámetros físicos a través de mediciones ópticas del índice de refracción.
		
	\begin{itemize}
		\item \textit{\textbf{Redes de difracción de Bragg \textit{(FBG - Fiber Bragg Grating)}}}	
		
		\begin{figure}[H]
			\centering
			\includegraphics[width=0.85\textwidth]{./img/operacionFBG}
			\caption{Funcionamiento de un sensor de fibra óptica FBG \cite{funcionamientoFBG}} 
			\label{fig:funcionamientoFBG}
		\end{figure}		
		
		Los sensores de fibra óptica basados en redes de difracción de Bragg (\textit{Fibre Brag Gratings}), también denominadas FBGs, se diseñan para reflejar determinadas longitudes de onda de la luz y transmitir el resto (Figura \ref{fig:funcionamientoFBG}). Para ello en el núcleo de la fibra se crea una variación periódica del índice de refracción, conocido cómo rejilla tipo Bragg. Esta variación sólo afecta a la transmisión de cierta longitud de onda, la que refleja. Este tipo de fenómeno se puede utilizar cómo filtro bloqueador de una longitud de onda, además de para medir parámetros físicos cómo la deformación o la temperatura. Por ejemplo, es muy común su uso para la monitorización de estructuras como puentes \cite{FOSensorFrancis}. 
		
		\begin{figure}[H]
			\centering
			\includegraphics[width=0.75\textwidth]{./img/FBGmanufactur}
			\caption{Proceso de generación del sensor de Bragg en la fibra. \cite{tesisUPMmalte}} 
			\label{fig:manufacturaBragg}
		\end{figure}
		
		 Para conseguir en el núcleo la variación permanente del índice de refracción se utiliza una fuente de luz ultravioleta (UV). De esta manera se inscribe una rejilla tipo Bragg en una fibra monomodo (figura \ref{fig:manufacturaBragg}). Comúnmente se utiliza fibra de sílice dopada con germanio, por su fotosensibilidad (capacidad de cambio del índice de refracción del núcleo con la exposición a la luz UV). En función de la intensidad y la duración de la exposición y de la fotosensibilidad de la fibra se consigue una variación del índice de refracción mayor o menor. 
		
		En resumen, los sensores de fibra de Bragg consisten en una fibra óptica monomodo, dónde, en un segmento reducido de esta, se encuentra una rejilla tipo Bragg. Siendo esta la que genera en el núcleo de la fibra el cambio periódico de índice de refracción \cite{defFBG}.\\
		
		Existe una relación matemática entre la longitud característica de la FBG o longitud de la onda reflejada ($\lambda\textsubscript{B}$), el índice de refracción efectivo ($\eta\textsubscript{eff}$) y el periodo de la red de Bragg ($\Lambda$), como se puede ver en la ecuación \ref{eq:condicBragg}, dónde se define la longittud de la onda reflejada, $\lambda\textsubscript{B}$: 
			\begin{equation}
				\label{eq:condicBragg}
				\lambda=\lambda\textsubscript{B} = 2\eta\textsubscript{eff}\Lambda	
			\end{equation}
				
		A partir de estos conocimientos se deduce cómo funciona la medición de deformaciones. Cuando se genera una deformación de la fibra y cambia la distancia entre las rejillas de Bragg, se genera una variación del índice de refracción al variar el periodo de la red de Bragg. Es decir, al deformarse la FBG se tiene una $\Lambda$ diferente respecto a la de reposo, cuando no se genera ninguna deformación. En el caso de las variaciones de temperatura generan un cambio de índice de refracción del silicio, debido al efecto termoóptico \cite{termoDeformFBG}. 
		
		\begin{figure}[H]
			\centering
			\includegraphics[width=0.85\textwidth]{./img/medicBragg}
			\caption{Representación esquemática de un sistema de  medición de deformaciones mediante fibras ópticas con redes de Bragg y analizador espectral óptico. \cite{tesisUPMmalte}} 
			\label{fig:medicBragg}
		\end{figure}
		
		Para utilizar las FBGs cómo sensores se puede construir una distribución como la de la figura \ref{fig:medicBragg}. En primer lugar, se manda un pulso a través de la fibra, y se pone un analizador de espectros para anotar las variaciones en la longitud de onda de Bragg ($\lambda\textsubscript{B}$). Según sea el escenario pueden ser necesarias más de una FBG. Para ello se puede realizar una configuración en serie (empalmando FBGs) o en paralelo (utilizando multiplexores). 
		
		\begin{figure}[H]
			\centering
			\includegraphics[width=0.65\textwidth]{./img/flexibleFBG}
			\caption{FBG embebida en un material flexible. \cite{nedomaPDMS}} 
			\label{fig:flexibleFBG}
		\end{figure}
	
		Además, para proteger la FBG es conveniente cubrirla con un material flexible (figura \ref{fig:flexibleFBG}), como por ejemplo el policloruro de vinilo (PVC) o el polidimetilsiloxano (PDMS). En este trabajo se utiliza como material protector y estructura del guante el PDMS, expuesto en el siguiente punto.
	
\end{itemize}


%--Desarrollo del prototipo
\section{Desarrollo del prototipo}
\label{sec:prototipo3}
%[Esta parte de desarrollo del proyecto parte de otro trabajo. Aquí mencionar algo que diga el trabajo de Silvia y mencionarla en la bibliografía.]

La realización del primer desarrollo se origina a partir de un trabajo comenzado con anterioridad en el grupo de investigación de algebra y aplicaciones del departamento de matemáticas de la universidad \cite{SilviaTFM}. 
El prototipo consiste en un prototipo técnico y funcional de un guante sensorizado para medir los ángulos de flexión de los nudillos y la muñeca.

Se aprovecha del prototipo anterior el cableado, la fuente y el interrogador. El guante anterior es un grosor demasiado fino, tiene una forma un poco deforme de mano y están los cables sin agrupar (siendo muy complicado su transporte)
Se mejora el soporte físico (hardware) volviendo a fabricar el guante y se desarrolla un nuevo programa con una interfaz de usuario simple e intuitivo. 


	\begin{figure}[H]
		\centering
		\includegraphics[width=0.85\textwidth]{./img/prototipo}
		\caption{Prototipo con Sensores de fibra FBG.} \label{fig:prototipoFBG}
	\end{figure}


 
\subsection{Materiales}
\label{sec:materiales3}
En este apartado se disponen brevemente los componentes utilizados para el desarrollo del prototipo. 


\begin{itemize}


\item \textbf{Fibras de Bragg}
\begin{figure}[H]
	\centering
	\includegraphics[width=0.60\textwidth]{./img/fibraBraggpaquete2}
	\caption{FBG de longitud de onda de 1536nm en su envoltura de compra.} \label{fig:FBGpaquete}
\end{figure}

Las fibras de rejilla de Bragg (FBG) empleadas son de la empresa Micron Optics (ver figura \ref{fig:FBGpaquete}). Consisten en un FBG centrado en una óptica recubierta don poliimida de dos metros de longitud. Se emplean las siguientes longitudes de onda para cada FBG según la flexión a medir:

\begin{table}[H]
	\centering
	\begin{tabular}[t]{|r|c|}
		\hline
		& Longitud de onda del sensor\\
		\hline
		\hline
		Dedo pulgar & 1532 nm \\
		\hline
		Dedo índice & 1548 nm \\
		\hline
		Dedo corazón & 1576 nm \\
		\hline
		Dedo anular & 1568 nm \\
		\hline
		Dedo meñique & 1560 nm \\
		\hline
		Muñeca & 1541.26 nm \\
		\hline
	\end{tabular}
	\caption{Tabla longitud de cada sensor FBG}
	\label{tabla:mmedidas 80 cm}
\end{table}

Puesto que las fibras de Bragg resultan muy sensibles ante roturas y la aplicación del prototipo precisamente consiste en contorsionar las fibras es necesario embeber las fibras en un material moldeable pero rígido. Para ello se emplea PMDS.
 
 \clearpage

	\item \textbf{Polidimetilsiloxano (\textit{PDMS})}
	
	Dentro de la familia de los polímeros orgánicos basados en silicio se encuentra el polidimetilsiloxano, también conocido como PDMS. Otro término por el que se le conoce es dimeticona, un tipo de aceite de silicona. Para abreviar, a partir de este punto en el documento se referirá a él como PDMS. El polidimetilsiloxano es un material cristalino, flexible y fácil de modelar \cite{propPDMS}.
	
	\begin{figure}[H]
		\centering
		\includegraphics[width=0.45\textwidth]{./img/compPDMS}
		\caption{Composición química del PDMS. \cite{nedomaPDMS}} 
		\label{fig:compFBG}
	\end{figure}
	
	Su formulación química es $(H_{3}C)_{3}SiO[Si(CH_{3})_{2}O]_{n}Si(CH_{3})_{3}$ (figura \ref{fig:compFBG}), siendo $[Si(CH_{3})_{2}O]_{n}$ el monómero presente n veces en la molécula del PDMS según la proporción de este con el agente de curación \cite{propPDMS}. \\ 
	Para su fabricación se mezcla el monómero con el agente de curación siguiendo una proporción n:1, en función de la consistencia con la que se desee el elastómero resultante. Cuanto mayor sea la n, mayor será la solidez del PDMS. Una vez hecha la mezcla es necesario que esta cure, ya sea a temperatura ambiente o aplicándole calor para que cure más rápidamente.
	
	\begin{figure}[H]
		\centering
		\includegraphics[width=0.49\textwidth]{./img/liquidPDMS}
		\includegraphics[width=0.49\textwidth]{./img/solidPDMS} 
		\\(a)\hspace{7cm}(b)
		\caption{PDMS en estado líquido(a) \cite{liquidoPDMS} y sólido(b) trás la curación \cite{solidoPDMS}} 
		\label{fig:slPDSM2}
	\end{figure}
	
	La utilización de este elastómero en el ámbito de este trabajo es una buena opción ya que es inofensivo, no tóxico, no inflamable y eléctricamente no conductor.\cite{nedomaPDMS}
	



\item \textbf{Divisor de fibra óptica}

Para que llegue el pulso transmitido a cada una de las fibras de Bragg hace falta un dispositivo que se capaz de dejar pasar la luz de una sola fibra a varias y combinar en el sentido inverso los pulsos de luz que retornen. Esta función la cumple un divisor de fibra óptica. Se ha empleado un splitter de PLC de una entrada y ocho salidas terminadas en conectores de férula, de las cuales sólo serán necesarias seis (ver figura \ref{fig:splitter}).

\begin{figure}[H]
	\centering
	\includegraphics[width=0.75\textwidth]{./img/splitter1a8}
	\caption{Splitter de PLC de 1:8} \label{fig:splitter}
\end{figure}

En cuanto a la transmisión de la luz y su posterior recepción se necesita una fuente y un interrogador. 

\item \textbf{Fuente}

La fuente es un emisor de luz de banda ancha que puede provenir de un diodo superluminiscente (SLED) o de emisión espontánea amplificada (ASE).  La fuente integrada en el prototipo es la fuente de diodo superluminiscente (SLED) DL-BP1-1501A de Ibsen Photonics (ver figura \ref{fig:FuenteInterrogador}). Esta fuente es comunmente utilizada para sistemas que utilizan FBGs. Emite pulsos de 70nm de ancho de banda en el rango de los 1550nm. La versión empleada dispone de una interfaz de LabView que permite configurar sus parámetros de emisión cuando fuera necesario. La fuente necesita ser configurada una vez y funciona según su última configuración la próxima vez que se encienda. 


\item \textbf{Interrogador}


El interrogador es el dispositivo que detecta la señal de luz proveniente de la fibra y, en este escenario, la transmite al ordenador. En este proyecto se trabaja con el interrogador USB I–MON 256/512 del proveedor Ibsen Photonics, al igual que la fuente (ver figura \ref{fig:FuenteInterrogador}). Este interrogador incluye un software implementado en LabView que sirve de herramienta para el usuario para visualizar la detección de luz desde el ordenador a través del puerto USB en tiempo real.  Este software sirve de base para la realización de la interfaz de este proyecto. 

\begin{figure}[H]
	\centering
	\includegraphics[width=0.45\textwidth]{./img/fuente}
	\includegraphics[width=0.45\textwidth]{./img/interrogador} 
	\\(a)\hspace{7cm}(b)
	\caption{(a) Fuente \cite{fuente} (b) Interrogador \cite{interrogador}} 
	\label{fig:FuenteInterrogador}
\end{figure}


 
\end{itemize}

Además se han diseñado y fabricado el molde para fabricar el guante y  un armazón para almacenar de manera organizada y segura el cableado de las fibras junto con el splitter.




\subsection{Fabricación del prototipo}
\label{sec:proceso3}
%Elaboración
 
 
Para que sea más cómoda la explicación del proceso de elaboración del prototipo se divide este en tres partes: modelado 3D, fabricación del guante y montaje del prototipo completo.

\begin{itemize}
	
% _ Modelado 3D:

	\item \textbf{Modelado 3D - Configuración 3D}
	
	Esta parte comprende el diseño del molde con el que se fabrica el guante y de la caja contenedora de todo el cableado. El software utilizado para el modelado 3D ha sido la versión de escritorio de SketchUp \cite{SketchUp}.
	
	
	Para poder producir el guante de PDMS es necesario tener un molde donde verter la disolución para darle la forma deseada. Gracias a las versatilidad de diseño que ofrece la impresión 3D se realiza con este proceso de manufactura el molde (véase figura \ref{fig:molde}). 

	\begin{figure}[H]
		\centering
		\includegraphics[width=0.6\textwidth]{./img/molde}
		\caption{Molde} \label{fig:molde}
	\end{figure}
 
 	El diseño 3D del molde se ha realizado a partir de la silueta de la huella de la mano. El recipiente con forma de mano necesita seis franjas a la altura de la muñeca para conducir por ellas las fibras de Bragg hasta el resto del prototipo. También tiene unas pequeñas muescas que sirven de guía para cuando se introduzcan la fibras en el PDMS.
 	 	
 	 	
 	En cuanto al diseño de la caja se realiza con la finalidad de contener en ella todos los cables de fibra óptica para tener un prototipo más ordenado y compacto. En la figura \ref{fig:ordenCaja} se observa la mejora que aporta el uso de esta caja, pasando de tener todas las fibras desordenadas y ocupando un espacio considerable a tenerlo todo ordenado. Además, esto protege el cableado ante roturas en traslados u otras situaciones. 
 	
	\begin{figure}[H]
	 	\centering
	 	\includegraphics[width=1\textwidth]{./img/ordenCaja}
	 	\caption{Antes y después de tener la caja} \label{fig:ordenCaja}
	\end{figure}




  	\begin{figure}[H]
		\centering
		\includegraphics[width=0.8\textwidth]{./img/caja}
		\caption{Diseño 3D de la caja} \label{fig:caja}
	\end{figure}
 	
 	El diseño se ha dividido en cuatro partes, ya que no cabía todo el diseño en las impresoras 3D disponibles. Las medidas completas son de $31.5\times15.5\times9$ cm. Tres de estas partes forman la base de la caja y una cuarta parte sirve de tapa protectora del compartimento de los conectores. La base se divide en el compartimento para los cables de fibra procedentes del guante, el compartimento para los conectores  y el del splitter. Las cuatro partes se encajan entre ellas gracias a unas ranuras incluidas en el diseño para este fin. Todo esto se puede observar en la figura \ref{fig:caja}.
 

 	Para la manufacturación del molde y del armazón se ha empleado tecnología de impresión 3D en material plástico PLA. Se han utilizado dos modelos de impresora debido a la disponibilidad de estas en la universidad, la impresora Hephestos de BQ y Witwox de BQ también (ver figura \ref{fig:impresoras3D}). Son muchos los materiales con los que se pueden realizar impresiones en 3D a partir de un diseño. En este caso ambos diseños se imprimen en PLA por ser biodegrable. Se trata de un polímero constituido por moléculas de ácido láctico, obtenido generalmente de tratar almidón de maíz, yuca o caña de azúcar. Además tiene precio y características adecuadas para las necesidades del proyecto. \cite{bioPLA}
 	
	 \begin{figure}[H]
	 	\centering
	 	\includegraphics[width=0.49\textwidth]{./img/hephestos}
	 	\includegraphics[width=0.49\textwidth]{./img/witbox} 
	 	\\(a)\hspace{7cm}(b)
	 	\caption{(a) Hephestos (b) Witbox} 
	 	\label{fig:impresoras3D}
	 \end{figure}
	
	 	
 	

% _ Fabricación del guante:
 
	\item \textbf{Fabricación del guante}

	Para el proceso de fabricación del guante se necesita el elastómero y el agente de cura que componen el PDMS. El PDMS utilizado corresponde a PDMS SYLGARD$^{\textregistered}$ 184 de Dow Corning (figura \ref{fig:pdms}).
			
	\begin{figure}[H]
		\centering
		\includegraphics[width=0.5\textwidth]{./img/PDMS}
		\caption{PDMS: Elastómero y agente de cura.} \label{fig:pdms}
	\end{figure}	

	

	Además es necesario el molde presentado en el apartado anterior y las fibras de Bragg. En este apartado se utilizan diversas herramientas del laboratorio: capsula petri, báscula, vaso de precipitados, pipeta de plástico, espátula y  horno. 
	

	A continuación, se procede a describir por pasos el proceso de fabricación del guante:
	\begin{enumerate}
		\item \underline{Moldeado PDMS}
		
		La base de la fabricación del guante es la mezcla a realizar para la obtención de la densidad del PDMS requerida. Si se tomase una disolución muy diluida el guante no sería lo suficientemente consistente. 
		
		
		
		\begin{figure}[H]
			\centering
			\includegraphics[width=1\textwidth]{./img/fabricacionGuante}
			\caption{Proceso de manufactura del guante.} \label{fig:fabricacionGuante}
		\end{figure}
		
		La mezcla de PDMS realiza tiene una relación agente-polímero de 1:10. Se coloca capsula petri sobre la báscula y se vierte en ella desde el vaso de precipitados 55g de elastómero, figura \ref{fig:fabricacionGuante}  [A]. Una vez se tiene la cantidad necesaria de polímero, se tabula la báscula y se toma el agente de curación desde el bote con ayuda de una pipeta de plástico, figura \ref{fig:fabricacionGuante} [B]. Se añade al elastómero 5.5g de agente de curación. Una vez se tienen las cantidades expuestas, se mezclan durante cuatro minutos con la ayuda de la espátula, figura \ref{fig:fabricacionGuante} [C]. Después de este paso en algunas ocasiones se generan burbujas de aire, que se pueden eliminar en un horno de vacío en pasos posteriores. En esta ocasión no ha sido necesario. Cuando se tiene una mezcla homogénea (se ha removido durante cuatro minutos correctamente) se vierte el PDMS en el molde del guante, figura \ref{fig:fabricacionGuante} [D].  	
		
		
		

		\item \underline{Colocación de los sensores FBG}
		
		\begin{figure}[H]
			\centering
			\includegraphics[width=0.8\textwidth]{./img/diagramaMano2}
			\caption{Disposición del guante sensorizado.} \label{fig:manoFBG}
		\end{figure}
		
		En este paso se colocan las fibras dentro del PDMS. En la figura \ref{fig:manoFBG} está representada la distribución de las fibras de Bragg en el PDMS frente a la colocación que se ha realizado. Aprovechando las franjas de la pared del molde y las guías de la base se han colocado las fibras dentro de la mezcla. Para que el trozo de fibra que contiene las rejillas esté en los ejes de giro de la mano (los vértices de los ángulos a medir) ha sido necesario cortar las fibras ya que el extremo superior no era necesario conectarlo posteriormente a nada. El otro extremo de la fibra se ha conducido por las franjas del molde hacia el exterior de este de manera ordenada.
		
	
		\item \underline{Curación PDMS} 
		
	
		Para que el PDMS se cure una opción es dejarlo al aire libre durante un par de días. La segunda opción y a más aconsejable para que la curación sea más rápida y completa se realice en el horno. 
		%Esta afirmación se solventa en la experiencia ya que durante la realización del proyecto ha habido que dejar el PDMS curándose al aire libre, y después de tres días aún estaba el PDMS sin curar del todo. Además, cuando el PDMS estuvo curado tenía una textura más pegajosa que al utilizar el horno. 
		Es por ello que se ha metido el molde con el PDMS y las fibras de Bragg en el horno durante cinco horas y media a $-55\,^{\circ}\mathrm{C}$). Una vez pasado ese tiempo se deja reposar una hora dentro del horno apagado. 
		
		\clearpage
		
 
		\item \underline{Desmoldar}
		
		\begin{figure}[H]
			\centering
			\includegraphics[width=0.95\textwidth]{./img/fabricacionGuante2}
			\caption{Desmoldado del guante.} \label{fig:fabricacionGuante2}
		\end{figure}
	
		Para finalizar el proceso de moldeado del PDMS, hace falta retirar del molde el PDMS con los sensores embebidos. Este paso hay que realizarlo con mucha paciencia y cuidado, ya que el PDMS está adherido al molde y las fibras de FBG son muy delicadas. Para ello se ha empleado la ayuda de la espátula. En la figura \ref{fig:fabricacionGuante2} se representa este paso.
		

	\end{enumerate}
	
%\underline{\textit{Nota:}} Cabe mencionar que durante la realización del proyecto este proceso ha sido necesario realizarse varias veces debido a los errores que han sucedido. Se detalla mejor en el apartado de resultados y conclusiones.
		


% _ Montaje completo:	

	
	\item \textbf{Montaje completo}
	
	Una vez se tiene la estructura del guante fabricada, hace falta proceder al acoplo de este al sistema completo. 
	
	En este apartado se utilizan varias herramientas del laboratorio: cortadora de precisión, alcohol, servilletas específicas para limpiar fibra, empalmadora, mechero y empalmes por fusión.
	
	%El conexionado entre el guante y los demás elementos que componen el prototipo se realiza con cables de fibra óptica monomodo unidos a través de conectores o empalmes. Los empalmes se han realizado mediante fusión con protección termocontraible. Los conectores utilizados para la fibra son los de Férula (\textit{Ferule Connector - FC}).
	
	\begin{figure}[H]
		\centering
		\includegraphics[width=1\textwidth]{./img/diagramaFBG}
		\caption{Diagrama de conexiones de elementos del prototipo.} \label{fig:diagramaFBG}
	\end{figure}
	
	La figura \ref{fig:diagramaFBG} representa la distribución del montaje completo del prototipo. El guante compuesto por seis fibras de Bragg (una por cada dedo y la muñeca) embebidas en PDMS. El guante realizado en el apartado anterior tiene la forma de huella de mano hasta la muñeca. De donde salen las seis fibras de Bragg hasta su empalme con la fibra óptica monomodo (en amarillo). Estas sirven de enlace con el multiplexor, teniendo un extremo empalmado a los sensores y el otro terminado en conectores FC. Gracias a ello, el multiplexor se conecta en el extremo de ocho fibras a la parte del sensor (dejando dos fibras en desuso) y del extremo de una única salida a la fuente. Las fibras del multiplexor están representadas en color azul. La fuente, alimentada además por la red eléctrica, está conectada al interrogador (representado en color amarillo). Siendo este último el que se conecte al ordenador por USB para poder obtener finalmente los datos medidos. Además, notese que en la figura \ref{fig:diagramaFBG} también se representa que parte de las fibras monomodo más cercanas a los sensores y el splitter se encuentra dentro de la caja impresa en 3D. 
	
			\clearpage
	
	 
	Posteriormente se toman las fibras monomodo y el splitter y se procede a la unión de estas tres partes. 
	
	El proceso de fusión (Figura \ref{fig:procFusion}) consiste en primero pelar, limpiar y cortar los extremos de las fibras que se desee unir. Una vez se dispone en los extremos de la fibra desnuda, cortada con la cortadora de precisión y limpia con un paño adecuado y alcohol, se deben colocar cada una en las guías de la fusionadora. Después se ejecuta el programa de fusión adecuado, según sea la fibra. Por último, se le coloca a la fusión un manguito protector termocontraible o una protección tipo mordaza para que la fusión no se desprenda ante alguna adversidad. 
	
	\begin{figure}[H]
		\centering
		\includegraphics[width=0.85\textwidth]{./img/procesoFusion}
		\caption{Proceso de empalme por fusión. \cite{FOAconect}} 
		\label{fig:procFusion}
	\end{figure}
	
	
	\begin{enumerate}
		\item \underline{Unión fibras FBG - Fibra monomodo}
		
	\begin{figure}[H]
		\centering
		\includegraphics[width=0.5\textwidth]{./img/union1}
		\caption{Unión fibras FBG - Fibra monomodo.} 
		\label{fig:union1}
	\end{figure}  
	
	Esta unión se realiza mediante fusionado de fibra, ya explicado. %Merece mención el paso de pelar la fibra de Bragg, ya que debido a lo fina que es ha sido necesaria la ayuda del mechero para pelarla. Además, para las demás herramientas de las que se disponía, la fibra desnuda también ha resultado más fina de para lo que estaban preparadas. En cual quier caso, tras varios intentos fallidos, se ha conseguido soldar las fibras debidamente. 
	Por último, es necesario colocar los protectores termocontraibles introducidos en la fibra antes de soldarla.
	
	\begin{figure}[H]
		\centering
		\includegraphics[width=0.95\textwidth]{./img/fusionFOpractica}
		\caption{Colocación de la fibra en la fusionadora y proceso por pantalla.} 
		\label{fig:fusionadora}
	\end{figure}  
	
	En la figura \ref{fig:fusionadora} se pueden observar dentro del proceso de empalme el paso de la colocación de la fibra en la fusionadora y su representación en la pantalla. %Sobre el proceso en pantalla, 
	Cuando la fibra está colocada en la fusionadora y cerrada la tapa, en la pantalla se ven aumentados los extremos de la fibra. La empalmadora se encarga de alinear las fibras para que los núcleos coincidan. Cuando la fusión se ha realizado, en la pantalla se muestra las pérdidas que tiene el enlace.  
	

	
	\item \underline{Fibra monomodo - Splitter}
	
	\begin{figure}[H]
		\centering
		\includegraphics[width=0.5\textwidth]{./img/union2}
		\caption{Fibra monomodo - Splitter.} 
		\label{fig:union2}
	\end{figure}  
	
	Para unir la fibra monomodo al splitter se emplean conectores de férula. Al ser ambos extremos a conectar de tipo macho se emplea un adaptador hembra-hembra. 
	
	\begin{figure}[H]
		\centering
		\includegraphics[width=0.85\textwidth]{./img/cajaOrden}
		\caption{Organización de las fibras dentro de la caja.} 
		\label{fig:CajaOrden}
	\end{figure}  
	
	Estas uniones se colocan en la caja (ver figura \ref{fig:CajaOrden}). Por un extremo de la caja se insertan las seis fibras monomodo al primer apartamento, almacena el cableado sobrante y en el segundo departamento se colocan las conexiones al splitter. Además, estas conexiones están protegidas por la tapa de dicho departamento. Por último, en el tercer departamento se recogen las ocho fibras del splitter salientes de las conexiones hasta llegar al mecanismo del splitter que reposa sobre la sección diseñada para ello. Desde este se accede a la perforación establecida en la pared, dando salida a la fibra aislada del splitter.

	
	Para una mejor organización, se han etiquetado debidamente todas las fibras. Lo que facilita posibles modificaciones futuras que vaya a ser necesario realizar al conexionado. 
		
	
	\item \underline{Conexión de la fuente y del interrogador}
	
	Completando el conexionado del prototipo la fibra aislada del splitter es conectada a la fuente. La fuente se une al interrogador a través de una fibra monomodo y el interrogador al ordenador. 
	
	En este apartado se tienen dos conexiones ópticas y dos eléctricas. Las conexiones eléctricas sirven en este escenario para alimentar la fuente y como medio de conexión entre el sistema óptico y el usuario a través del ordenador. 
	
	Tanto el interrogador como la fuente disponen de un software en LabVIEW$^{\textregistered}$. El software de la fuente tiene como funcionalidad configurar la señal óptica que ha de emitir la fuente. Por otro lado, el software del interrogador sirve para interpretar la señal recibida en este.
	
		
	\end{enumerate}
	 
	
\end{itemize}

En este punto de la memoria se encuentran todos los elementos del prototipo dispuestos para su funcionamiento.



\subsection{Programación del software}
\label{sec:programacion}

En este apartado se presenta el desarrollo del software para la obtención de los datos. Toda la programación software se ha realizado en LabVIEW$^{\textregistered}$ ya que el software parte del programa del interrogador como base de programación. 

\begin{itemize}		
%-- LABVIEW 	
	
	\item \textbf{LabVIEW (\textit{\underline{Lab}oratory \underline{V}irtual \underline{I}nstrument \underline{E}ngineering \underline{W}orkbech})}


	\begin{figure}[H]
		\centering
		\includegraphics[width=0.45\textwidth]{./img/LabVIEWicon}
		\caption{Logo LabVIEW. }
		\label{fig:LabVIEWicon}
	\end{figure}


	LabVIEW es un software de National Instruments de ingeniería que pretende simplificar el diseño de sistemas software distribuidos de pruebas, medidas y control \cite{LabVIEWpage}. Es un lenguaje y un entorno de programación gráfica desarrollado por National Instruments. \\	
	En sus inicios LabVIEW estaba orientado únicamente a aplicaciones de control de equipos electrónicos utilizados para el desarrollo de sistemas de instrumentación. Tiene dos ventanas principales: el \textbf{Panel frontal} y el \textbf{Diagrama de bloques}, como se ve en la figura \ref{fig:ejLabVIEW}. El panel frontal alberga los botones, pantallas, etc. con el que el usuario interactúa una vez está desarrollado el software, interfaz de usuario. Mientras que el diagrama de bloques corresponde a la circuitería interna del programa, dónde se interconectan los elementos del panel frontal para operar con ellos, dando lugar a la programación del backend del software \cite{LabVIEWbook}. 

\begin{figure}[H]
	\centering
	\includegraphics[width=0.95\textwidth]{./img/LabVIEWej}
	\\\hspace{2.5cm}(a)\hspace{6.5cm}(b)
	\caption{(Ejemplo sencillo de LabVIEW: (a) Frontend; (b) Backend}\cite{LabVIEWyt} 
	\label{fig:ejLabVIEW}
\end{figure}

	En LabVIEW la programación se realiza en el Diagrama de bloques. Los programas están compuestos por los siguientes elementos: controles, funciones e indicadores. Estos elementos se conectan mediante ``cables''. De esta manera se genera el programa, definiendo la ``circulación'' de los datos.  

\end{itemize}

El desarrollo del software se divide en los hitos de ``Familiarización con el programa'', ``Establecimiento del protocolo de medida'', ``Interfaz de usuario'', ``Generación de informes-Guardado'' y ``Resultado final del software''. Todos estos hitos se exponen a continuación:

		\clearpage


\begin{itemize} [label=]
	\item \textbf{Hito 1:} \textit{\textbf{Familiarización con el programa}}
	
	Puesto que se tiene el programa del interrogador que sirve para analizar la respuesta del sistema se ha de partir por entender el funcionamiento del software. De esta manera se identifican los puntos de la programación en los que se han de realizar los cambios o añadidos. 
	
	Una vez identificado el archivo VI a modificar, se muestra en la figura \ref{fig:hierarchy} los VIs con los que tienen dependencia (el circulo rojo rodea el VI de la interfaz del software y las lineas azules corresponden a las dependencias directas).
	
	\begin{figure}[H]
		\centering
		\includegraphics[width=0.85\textwidth]{./img/hierarchyLV}
		\caption{Árbol jerárquico de VI del proyecto.} 
		\label{fig:hierarchy}
	\end{figure}  
	
	Partiendo de este programa se han realizado los cambios necesarios para desarrollar el software de este proyecto.
	
	%y se han añadido los convenientes. Pero la cuestión es que hace falta determinar, una vez se comprenden las posibilidades del software, el protocolo que se va a seguir en la toma de medidas para desarrollar el software del prototipo. 
	
	\begin{figure}[H]
		\centering
		\includegraphics[width=1\textwidth]{./img/softwareInterrogador2}
		\caption{Interfaz del programa del interrogador.} 
		\label{fig:interrogadorPantalla}
	\end{figure}

	La figura \ref{fig:interrogadorPantalla} muestra las cuatro ventanas que compone el VI sobre el que se ha trabajado (previa a su manipulación). Estas cuatro ventanas corresponden a las ventanas de: gráfica de espectro, gráfica de longitud de onda, gráfica de FFT y configuración. Además, ofrece la posibilidad de guardar las medidas en tres formatos diferentes. Se puede observar como este tipo de interfaz es muy útil para analizar cualquier escenario que se desee desde el interrogador. Pero no resulta una buena opción para el tipo de escenario en el que se pretende utilizar el prototipo ya que se trata de un escenario muy concreto, con unos parámetros a medir específicos.
	
		Después de probar con diferentes valores, se ha tomado la decisión de mantener los parámetros de configuración del interrogador constantes con los siguientes valores:
	
	\begin{table}[H]
		\centering
		\begin{tabular}{|
				>{\columncolor[HTML]{EFEFEF}}l |c|}
			\hline
			Frecuencia de muestreo máximo			& 		$1000\; Hz$			\\ \hline
			Unidad de medida del eje x				& 		$nm$				\\ \hline
			Tiempo de exposición 					& 		$80\;\mu m$			\\ \hline
			Compensación por temperatura 			& 		$Desactivado$		\\ \hline
			Threshold								& 		$4\;\%$				\\ \hline
			``Fit type'' 							& 		$Gauss$				\\ \hline
			``Max number of FBG's'' 				& 		$6$					\\ \hline
			``Circular sample buffer length''       & 		$501$				\\ \hline
			``Wavelength/FFT graph buffer length''	& 		$501$				\\ \hline
			``N$^{\circ}$ of lines unified''		& 		$10$				\\ \hline		
		\end{tabular}
		\caption{Valores fijos de configuración del interrogador}
		\label{tabla:configuracionInterrogador}
	\end{table}	
	
	De esta manera se simplifica el ejercicio de configuración al usuario final. Es necesario generar una nuevo interfaz de usuario dónde se plasmen de manera intuitiva el significado físico de los valores obtenidos gracias al prototipo. Es por esta razón que ha sido necesario establecer un protocolo de medida.
	
	\item \textbf{Hito 2:} \textit{\textbf{Establecimiento del protocolo de medida}}
		
	El protocolo de medida está limitado por varias cuestiones. Ha sido necesario realizar una interfaz de usuario intuitivo para el usuario final, el fisioterapeuta o profesional sanitario. % un profesional fisioterapeuta, cuyo conocimiento sobre la tecnología empleada y su funcionamiento es limitado. 
	
	
	
	A partir de estas restricciones se refleja en el software el procedimiento de medida que permitirá que el software se ajuste a las necesidades físicas del guante. Se decide generar el siguiente protocolo de medida que aparece en el anexo \ref{sec:anexo1}.

	 
	
	\item \textbf{Hito 3:} \textit{Interfaz de usuario}
	
	\begin{figure}[H]
		\centering
		\includegraphics[width=1\textwidth]{./img/interfazSMinicio}
		\caption{Interfaz del programa SENSMOV en LabVIEW.}
		\label{fig:interfazinicio}
	\end{figure}
	 
	Uno de los retos de este proyecto es conseguir representar los datos medidos de manera visual. Para ello se ha diseñado una interfaz de usuario intuitivo. En la figura \ref{fig:interfazinicio} se muestra una captura de la interfaz de usuario cuando inicia el programa. En ella se pueden observar los pasos enumerados en el segundo hito (\textit{Establecimiento del protocolo de medida}).
	
	La interfaz de usuario se puede dividir en varios bloques para comprender mejor su funcionamiento:
	
	\begin{itemize} [label=]
		\item \underline{\textit{Bloque de interacción: }} El bloque de interacción entre el usuario y el software comprende los elementos que permiten al usuario comunicarse con el programa. Entre ellos se encuentra el cuadro de calibración, el cuadro de medición, el área de texto y los botones generales. 
		
		Para facilitar la experiencia del usuario, los botones de la aplicación se habilitan y desabilitan según el protocolo de medida. %De esta manera aunque el usuario se confunda o se despiste, es el propio programa el que le indica el paso siguiente. Esto sucede en el caso del cuadro de calibración y el de medición, es necesario que antes de comenzar la medición primero se ejecuten todas las calibraciones para que la visualización funcione correctamente. Sin embargo, el nombre del paciente puede introducirse en cualquier momento de la ejecución, ya que este en realidad no se utiliza hasta que se cierra el programa y se guarda el informe. 
		Los botones de \underline{\textit{INICIO}}, \underline{\textit{SALIR}} y \underline{\textit{ARCHIVAR RESULTADOS}} se pueden activar en cualquier momento de la ejecución.
		
		\item \underline{\textit{Bloque de visualización: }} El bloque de visualización está compuesto por dos ventanas: los indicadores del movimiento y la gráfica del espectro. Este último se utiliza como visualizador del comportamiento de las FBGs. Permite identificar en un rapido vistazo si el programa está funcionando correctamente. %Si no se identifican los picos debido a las FBGs significa que algo no está funcionando bien. El error puede ser que el interrogador no esté funcionando correctamente y haya que volver a ejecutar el software. 
		
		Los indicadores del movimiento consisten en seis deslizadores, uno por cada FBG. Los indicadores están programados para que durante el paso de medición muestren entre el valor de reposo y el de los máximos de extensión y flexión en que posición se encuentra cada dedo (y muñeca). Cada uno de los dedos tiene su propio rango de movimiento.  
		
		Como se observa en la figura \ref{fig:interfazinicio} cada indicador tiene un rango que va desde el porcentaje de -120 hasta 120. Las lineas horizontales cortan los valores de -100, 0 y 100. Estos valores están relacionados proporcionalmente con el máximo de extensión (100), flexión (-100) y reposo (0).
		Por lo tanto, cuando se establecen los máximos de extensión y flexión del paciente, se está determinando la relación con la posición de reposo del 100\% y -100\% respectivamente.
		
		
		%\item \underline{\textit{Bloque de elementos decorativos: }} Como elementos decorativos se tiene una barra superior y la huella de la mano. La huella de la mano sirve además como leyenda esquemática de los colores del indicador de movimiento.
		
		 	
	\end{itemize}
	
		
	\item \textbf{Hito 4:} \textit{Generación de informes-Guardado}
	
	Después de la sesión de rehabilitación, cuando el fisioterapeuta pulse \underline{\textit{ARCHIVAR RESULTADOS}} el software genera un archivo de texto dónde se plasman los resultados del ejercicio. El documento de texto generado se muestra en el anexo \ref{anexo2} que contiene los siguientes datos, distribuidos en 6 columnas, uno por cada articulación medida:
	

	Anteriormente se hace referencia al programa en LabVIEW que trae el interrogador. Este incluye tres opciones diferentes de guardado: ``\textit{Save Wave (Sample Buffer)}'', ``\textit{Save raw (Sample Buffer)}'' y ``\textit{Manual save in data log}''. Para la solución sel software diseñado se emplea como referencia este último, aunque aún y todo necesita de bastantes modificaciones ya que ``\textit{Manual save in data log}'' guarda únicamente la información de una ventana.
	
	\item \textbf{Hito 5:} \textit{Resultado final del software}

	El software diseñado ha de ser utilizado junto con el guante. Teóricamente los datos obtenidos servirían al fisioterapeuta para llevar un seguimiento del avance del paciente. Pero la realidad es que debido a los cambios que sufre la señal cuando el guante se coloca un poco diferente entre sesiones esto no es posible. Por lo tanto esta tecnología sirve para evaluar únicamente los resultados intersesiones.
	
	El desarrollo del software se ha llevado a cabo en paralelo al desarollo del hardware. El primero ha sido el que en muchos casos a limitado al desarrollo del segundo. En la figura \ref{fig:interfazSM} se puede ver en la gráfica cómo dos de las seis FBGs no están funcionando, lo que hace que el software no vaya a funcionar en su plenitud. 
		
	\begin{figure}[H]
		\centering
		\includegraphics[width=1\textwidth]{./img/interfazSM}
		\caption{Interfaz del programa de labview.}
		\label{fig:interfazSM}
	\end{figure}	
	
	En aspectos generales el software se ha diseñado según lo esperado. Obteniendo un resultado que se ajusta a las posibilidades del entorno.
\end{itemize}


\subsection{Funcionamiento}
\label{sec:funcionamiento3}

Este apartado expone como se coordina la configuración del soporte físico con el software programado, es decir, como funciona el prototipo en su plenitud. Para ello se referencia la figura \ref{fig:diagramaFBGfuncionamiento}, dónde cada cambio en la señal se encuentra dibujado y referenciado con un número.

\begin{figure}[H]
	\centering
	\includegraphics[width=0.95\textwidth]{./img/diagramaFBGfuncionamiento}
	\caption{Diagrama de funcionamiento del prototipo.} \label{fig:diagramaFBGfuncionamiento}
\end{figure}

La fuente, alimentada por una red de corriente alterna de $\sim$230V a 50Hz, emite un pulso de luz [1]. En el splitter este la señal se multiplexa de una a ocho fibras [2]. En el diagrama se representa lo que sucede en el caso del dedo corazón para simplificarlo, pero en las demás articulaciones el funcionamiento es análogo en diferentes longitudes de onda. Cuando el pulso de luz llega a la rejilla de Bragg, parte de este sigue su propagación hasta la punta del dedo dónde se termina por dispersar en el PDMS [3] y se refleja la otra parte [4]. La parte del pulso que se refleja se encuentra centrada a la longitud de onda característica del FBG. Este pulso se combina junto los otros pulsos de retorno de las otras cinco FBGs [5] llegando a la fuente. La señal óptica se transmite desde la fuente al interrogador por la fibra óptica [6]. En el interrogador la señal se convierte a señal eléctrica y se transmite a través del puerto USB al ordenador [7]. Siendo en este dónde será la señal procesada.

En el ordenador, dónde se es está ejecutando en el programa se visualizaría el espectro de la señal, los seis pulsos en sus diferentes longitudes de onda. Y con la información de estos pulsos es con la que el programa utiliza. 



%\section{Discusións}
%\label{sec:resultados3}


%El prototipo se ha finalizado. Han sido necesarios varios intentos para la manufacturación del guante. Lo cual ha ralentizado mucho el procedimiento, ya que se han llegado a tener que pasar por hasta cuatro modelados del PDMS. Probablemente esto sea debido a las posibilidades limitadas que brinda e laboratorio de la universidad y la falta de experiencia. Las fibras de Bragg son muy delicadas. El proceso de soldado también fue un punto problemático, ya que las fibras desnudas eran demasiado finas y el pelado se complicó. 

%Además, la fisionomía del guante no es lo más adecuado para el tipo de función que va a cumplir. Pero con las herramientas de las que se dispone en la universidad era lo más adecuado para su manufacturación. 

%Por otro lado, el software ha evolucionado desde la idea inicial hasta lo que ha terminado siendo. Esta evolución ha servido para que a pesar de las limitaciones existentes por la parte hardware el prototipo pudiera seguir teniendo utilidad.

%En conclusión, a pesar de que el prototipo no cumple con todas las expectativas se ha conseguido mejorar el prototipo físico de partida y diseñar un software intuitivo.

