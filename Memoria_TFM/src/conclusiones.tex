\chapter{Conclusiones y líneas futuras\label{sec:conclusiones}}


Este ejercicio resulta verdaderamente beneficioso en el aprendizaje. El comparar dos tecnologías diferentes para una misma aplicación permite entender en mayor magnitud las posibilidades de cada una de las tecnologías. 

\textcolor{rositaoscuro}{Los resultados de este trabajo han sido útiles para determinar la viabilidad al realizar un primer prototipo como solución al problema planteado.}

Este ha sido un trabajo muy completo al tomar tanto en cuenta ejercicio de desarrollo software como hardware. 

Se concluye el trabajo descartando la tecnología de sensores de FBG para la evaluación de la rehabilitación de las manos. Siendo esto motivado por no cumplir con las expectativas funcionales requeridas, además de por la complejidad de su manufacturación y su delicadeza. 

Se ha propuesto un diseño que sí que cumpliría con las espectativas de medir el movimiento de las manos, obteniendo datos comparables entre sesiones. Esta solución está planteada prestando especial atención a que vaya a cumplir con las expectativas que el otro prototipo no ha cumplido. Ofreciendo además mayor versatilidad y capacidad de evolución. Con los sensores inerciales es posible dotar al prototipo de características como la comunicación inalámbrica. Otras ventajas de la solución de sensores inerciales son la rosbusted del diseño y el bajo precio de este.

El software empleado en el desarrollo ha realentizado el desarrollo del prototipo (junto con los problemas de manufacturación encontrados). Y hace que pese a estar el prototipo dentro del marco de producto, no es este muy vistoso al usuario, y la experiencia de usuario se ve muy limitada. Por ello para el nuevo dispositivo la propuesta es programar la aplicación sobre C++, dejándole al desarrollador una mayor versatilidad en la programación.

\section{Líneas futuras}

Para poder desarrollar el prototipo con un mejor concepto ergonómico se aconseja en un futuro la colaboración con un profesional en el diseño de producto. 

Se plantea comenzar el desarrollo del guante con sensores inerciales propuesto por el desarrollo exclusivo del dedo pulgar junto con el índice. De esta manera se divide el escenario en varios escenarios, simplificando su resolución. Comenzando por tener dos dedos funcionales, se puede llegar a tener todas las articulaciones en funcionamiento y finalmente añadir la capacidad de comunicación inalámbrica. En paralelo a este desarrollo se realaizará el desarrollo del software por su dependencia, dejando para el final el desarrollo grueso de las funcionalidades más allá de la interpretación de las medidas de los sensores.

Una posibilidad más que ofrece el guante es añadirle funcionalidad actuadora para acompañar a los pacientes en el movimiento de las manos durante sus ejercicios de rehabilitación. 






 
