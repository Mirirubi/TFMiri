\chapter{Conclusiones y líneas futuras\label{sec:conclusiones}}

Se ha diseñado un sensor de ultrasonido para medir la distancia de separación entre piernas para una posterior implementación en un sistema que permita la medida de la distancia de separación entre pasos.

Una vez realizadas las distintas medidas de la distancia de separación, tanto en estático como en dinámico, puede concluirse que el sistema propuesto es adecuado para la determinar la distancia entre piernas de un sujeto al caminar.

Los resultados de este trabajo han sido útiles para determinar la viabilidad al realizar un primer prototipo como solución al problema planteado.

Con el fin de lograr una mayor precisión en las medidas, debido a que la velocidad del sonido depende de la temperatura según la ecuación \ref{eq:sound}, se propone como línea futura añadir un sensor de temperatura.
\begin{equation}\label{eq:sound}
V_{sonido} = 331.4 + 0.6*T_{c}
\end{equation}

Por otra parte, será necesario trabajar en el diseño de un sensor más compacto para mejorar su funcionalidad y conseguir un diseño más ergonómico.


Además, como línea futura se pretende crear un prototipo que pueda procesarse en tiempo real permitiendo al personal sanitario obtener información instantánea que ayude en la valoración de la evolución del paciente. Así, se deberán automatizar los algoritmos de los que se dispone actualmente y también desarrollar un software que permita la sincronización de los sensores inerciales con el de ultrasonidos para llevar a cabo el procesado on-line.

También se intentará hacer una futura validación del sistema comparando las medidas de la distancia entre pasos usando el sensor de ultrasonidos y las obtenidas con tapices de presión o instrumentalizados.

Por último, para poder realizar la medida objetivo de la distancia de separación entre pasos, será necesario será necesario incluir un algoritmo de corrección de deriva en la posición que permita determinar las distancias de los sensores inerciales de forma precisa. 
