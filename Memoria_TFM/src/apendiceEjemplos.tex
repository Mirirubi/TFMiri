\chapter{Algoritmos empleados\label{sec:ejemplos}}

%
% Breve guía de comandos útiles para la memoria
%



% Citar un elemento del glosario
Citamos el acrónimo \gls{FPGA}.

% Citar un elemento del glosario (primera letra en may´usculas)
\Gls{bitstream} es una secuencia de bits.

% Insertar una imagen con pie de página
\begin{figure}[H]
  \centering
  \includegraphics[width=0.14\textwidth,clip=true]{escudo}
  \caption{Logo de la Universidad Pública de Navarra.}
  \label{fig:logo_upna}
\end{figure} 

% Referenciar una etiqueta (label)
La figura~\ref{fig:logo_upna} se utiliza en la portada.

% Nueva página
\clearpage



% Fórmula dentro de una línea de texto
La ecuación de Euler ($e^{ \pm i\theta } = \cos \theta \pm i\sin \theta$) es citada frecuentemente como un ejemplo de belleza matemática.

% Fórmula independiente
\begin{equation}\label{eq:pythagoras}
a^2 + b^2 = c^2
\end{equation}


	
\section{Arduino}
	
		% Añadir código fuente sin líneas
		\begin{lstlisting}[label=algoritmo:Arduino,language=C,frame=single,caption=Algortimo en Arduino para obtención de distancia sensor ultrasonido]
	#include <SoftwareSerial.h> 
			
	SoftwareSerial BT(10,11);
			
	const int EchoPin = 5;
	const int TriggerPin = 6;
			
	void setup() {
		BT.begin(9600);
		pinMode(TriggerPin, OUTPUT);
		pinMode(EchoPin, INPUT);
	}	
	void loop() {
		int cm = ping(TriggerPin, EchoPin);
		BT.println(cm);
		delay(60);
	}
			
	int ping(int TriggerPin, int EchoPin) {
		long  distancia;
		long duration;
		digitalWrite(TriggerPin, LOW);  
		delayMicroseconds(4);
		digitalWrite(TriggerPin, HIGH); 
		delayMicroseconds(10);
		digitalWrite(TriggerPin, LOW);		
		duration = pulseIn(EchoPin, HIGH);  	
		distancia = duration * 10 / 292 / 2;  
		return distancia;			
}
		\end{lstlisting}
	