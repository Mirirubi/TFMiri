% Resumen en inglés
\chapter*{Resumen}

\begin{abstractEn} 
Human gait analysis is commonly used in rehabilitation, sport training and functional diagnosis. Ambulatory systems for assessing human movements must be portable, ergonomic and economically viable. 

The main aim of this work is to design an ultrasound based wireless measuring system to calculate step distances during walking.

The low-cost ultrasonic sensor developed allows us to calculate step distances. The distance has been achieved by merging data provided by our measurement system and by commercial inertial sensors. To ensure the best performance of the system, the accuracy of the ultrasonic sensor has been estimated.

Our results will be very useful in areas such as neurorehabilitation as they provide objective measurements of the rehabilitation and improvement degree in patients with stroke.



\end{abstractEn}

% Palabras clave en inglés
\begin{keywordsEn}
Inertial sensor, Ultrasonic sensor, Step distance measurement, Stroke.
\end{keywordsEn}

% Resumen en español
\chapter*{Resumen}

\begin{abstractEs}
 El análisis de la marcha es habitual en áreas como rehabilitación, entrenamiento deportivo y diagnóstico funcional. Los sistemas de medida ambulatorios deben ser portátiles, ergonómicos y de bajo coste.
 
 El objetivo principal de este trabajo es el diseño de un sistema de medida basado en la tecnología de ultrasonido para calcular la distancia de separación entre pasos durante la marcha.
 
 El sensor de ultrasonido desarrollado permite calcular dicha distancia combinando los datos del sensor de ultrasonido con los proporcionados por sensores inerciales comerciales. Para asegurar el correcto funcionamiento del sistema se ha evaluado la precisión del sensor de ultrasonidos diseñado.
 
 Los resultados serán de gran utilidad en el campo de la neuro-rehabilitación debido a que permiten obtener datos objetivos sobre el grado de mejora en pacientes que han sufrido un ictus.



\end{abstractEs}

% Palabras clave en español
\begin{keywordsEs}
Sensor inercial, Sensor ultrasonido, Distancia de paso, Ictus.
\end{keywordsEs}
