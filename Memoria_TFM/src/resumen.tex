% Resumen en inglés
\chapter*{Abstract}

\begin{abstractEn} 
%The number of survivors of neuronal diseases is higher due to the medical progress in the treatment. Consequently, the need to improve rehabilitation techniques is growing.
%The techonogies dedicated to measurement and operation are essential in the ae of neuro-rehabilitation.
%This work focuses on the measurement work. First, it is studied neurological diseases. Then, the interest is focused on hand rehabilitation.
%A fiber bragg grating bassed glove has been developed in this conception. In addition, an alternative solution based on inertial sensors has been designed.
The number of survivors after a cerebrovascular disease is higher due to the current medical progress in treatment. Consequently, the need to improve rehabilitation techniques that increase the quality of life of patients is growing.

The tools dedicated to measurement and monitoring are essential in neurorehabilitation, both for the patient and for the physiotherapist. This work focuses on the development of a prototype to measure the ranges of the hand joint after a cerebrovascular disease.

In this conceptual framework, a fiber optic glove has been developed, whose results and limitations have led to its rejection as a possible solution. Consequently, an alternative solution based on inertial sensors has been studied.



\end{abstractEn}

% Palabras clave en inglés
\begin{keywordsEn}
Inertial sensor, Fiber Bragg Gratting, Rehabilitation, Stroke.
\end{keywordsEn}

% Resumen en español
\chapter*{Resumen}

\begin{abstractEs} %Gracias al avance de la medicina en el tratamiento de las afecciones neuronales, es mayor el número de supervivientes. Lo que conlleva a una necesidad incremental de mejorar los procesos de recuperación.
%Dentro de la neuro-rehabilitación son fundamentales las tecnologías dedicadas a la medición y la actuación. 
%Este trabajo se centra en la tarea de la medición. En un primer lugar, se estudia el marco de aplicación de la tecnología,  las afecciones neurológicas. Posteriormente se canaliza el interés en la rehabilitación de las manos.
%En este marco se desarrolla un prototipo de guante medidor de la rehabilitación basado en sensores de FBG. Además, se diseña una solución alternativa basada en sensores inerciales.
Gracias a los grandes avances en la medicina actual, el número de supervivientes después de una accidente cerebrovascular cada vez es mayor. En consecuencia, la necesidad de mejorar las técnicas de rehabilitación que aumentan la calidad de vida de los pacientes está creciendo.

Las herramientas dedicadas a la medición y la monitorización son esenciales en la neurorrehabilitación, tanto para el paciente como para el fisioterapeuta.

Este trabajo se centra en el desarrollo de un prototipo para medir los rangos articulares de la mano y los dedos después de una enfermedad cerebrovascular.

En este marco conceptual, se ha desarrollado un guante de fibra óptica. Debido a sus limitaciones en el desarrollo y los resultados obtenidos, se ha descartado el guante de fibra óptica como una posible solución. En consecuencia, se ha estudiado una solución alternativa basada en sensores inerciales.


\end{abstractEs}

% Palabras clave en español
\begin{keywordsEs}
Sensor inercial, Sensor de fibra de Bragg, Rehabilitación, Ictus.
\end{keywordsEs}
