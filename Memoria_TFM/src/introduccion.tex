\chapter{Introducción}

Un accidente cerebrovascular (ACV), también conocido como ictus se debe o bien a una interrupción del suministro sanguíneo en el cerebro o, bien, a una hemorragia cerebral.
%justificación
El accidente cerebrovascular (ACV) es la enfermedad neurológica adquirida más común en la población adulta de todo el mundo. Gracias a los avances médicos, la mortalidad en torno a esta dolencia se ha visto mermada significativamente. Lo que conlleva un incremento en el número de discapacitados por esta afección. \cite{miradaRehab}

Según el estudio IBERICTUS (del Proyecto de Apoplejía del Grupo de Estudio de Enfermedades Cerebrovasculares) realizado en 2012 se tienen 187 casos de ictus por cada 100.000 habitantes al año. Entre las personas que sufren un ictus un 14\% fallecen y los otros 86\% sobreviven\cite{iberIctus}.  De entre los que sobreviven un 8\% son causa de un ataque isquémico transitorio (AIT), un 15\% de hemorragias y un 77\% de infartos. De entre os dos últimos casos (ictus consecuencia de hemorragias e infartos) un 44\% de las personas quedan con secuelas, es decir, necesitan rehabilitación. Este porcentaje se divide en un 23\% de casos de discapacidad leve-moderada y un 21\% de discapacidad grave. Las personas con discapacidad grave además de rehabilitación necesitan también atención constante.  \cite{iberIctus}

Cuánto antes se inicie la rehabilitación de este tipo de pacientes más probable será conseguir mejores resultados. Incluso será necesario un seguimiento posterior a la fase de recuperación. Mediante la rehabilitación se pretende minimizar el déficit neurológico y sus complicaciones, fomentar la reintegración familiar y social del individuo y mejorar su
calidad de vida. Con este fin, el equipo de rehabilitación debe usar todas las técnicas y tecnologías que han demostrado ser útiles para el manejo de estos pacientes en diferentes etapas de rehabilitación, incluso en el seguimiento posterior a la fase de rehabilitación. \cite{miradaRehab}

%En este marco se sitúa el prototipo desarrollado en este trabajo, dentro de este tipo de tecnologías que ayudan al equipo de rehabilitador en los ejercicios de rehabilitación. Favoreciendo la recuperación del paciente, que es lo que se persigue en esta situación.

Fruto de la colaboración entre la Universidad Pública de Navarra (UPNA) y la Asociación del Daño Cerebral de Navarra	(ADACEN) surge el proyecto ``\textit{SENSMOV}(Sistema TIC de evaluación de la movilidad para personas con problemas motores)'', financiado por el Departamento de Desarrollo Económico del Gobierno de Navarra. Se comenzó el primer prototipo de un guante de sensores de fibra óptica basado en redes de difracción de Bragg (FBG) para monitorizar los movimientos de la mano y dedos con el objetivo de utilizarse durante la rehabilitación del rango articular de la mano.

%Se parte de un 

El objetivo de este trabajo de fin de máster es continuar el desarrollo de dicho prototipo y proponer una solución alternativa que mejore sus especificaciones y funcionalidades. 



\section{Objetivos}
\label{sec:objetivos1}
 

\subsection{Objetivo principal}
\label{sec:objPrinc1}

El objetivo principal de este proyecto es mejorar un prototipo técnico y funcional de una guante destinado a cuantificar los ejercicios de rehabilitación en pacientes que han sido afectados por ACV. Así como un software que sea capaz de monitorizar los movimientos de las articulaciones de las manos. El prototipo deberá ser diseñado con la finalidad de evaluar la rehabilitación del rango articular de la mano en pacientes con movilidad reducida. Para ello se tendrán en cuenta carácterísticas como la ergonomía, precisión y fiabilidad del guante, y la usabilidad y funcionalidad del software. 

%Puesto que se engloba el proyecto dentro de un marco de producto, también será importante buscar un precio moderado de componentes.


\subsection{Objetivos específicos}
\label{sec:objEspec1}

%Este proyecto parte de un proyecto propuesto con sensores FBG ya comenzado en la Universidad, por lo tanto antes de idear una nueva solución, será necesario cerrar el desarrollo de la otra. Además 

A continuación se enumeran ciertos objetivos específicos en los que se puede dividir el trabajo para comprender su magnitud: 

\begin{enumerate}
	\item Familiarización con el prototipo diseñado (Hardware y software)
	\item Propuesta de mejora frente al viejo prototipo (Hardware y software)
	\item Desarrollo final del prototipo con la tecnología de FBGs (Hardware y software)
	\item Ideación de una solución alternativa que sea capaz de mejorar el resultado obtenido en el desarrollo: solución con sensores inerciales.
	
\end{enumerate}

Este trabajo servirá para comprender en plenitud el problema que resuelve el proyecto y las necesidades que conlleva.


\section{Estructura del documento}
\label{sec:disposicion1}

El documento consta de los siguientes capítulos: 

\begin{itemize}[label=]
	\item {\textbf{CAPÍTULO 1:} Introducción.}
	\item {\textbf{CAPÍTULO 2:} Estado del arte.}
	\item {\textbf{CAPÍTULO 3:} Solución con sensores de fibra FBG.}
	\item {\textbf{CAPÍTULO 4:} Solución con sensores IMU.}
	\item {\textbf{CAPÍTULO 5:} Conclusiones y líneas futuras.}
\end{itemize}


En el desarrollo del proyecto se ha llevado a cabo la solución con sensores de fibra FBG y propuesto (tras su estudio) una segunda solución basada en sensores IMU. En ambos casos, la exposición de la solución tomada se divide en la explicación teórica en el \textit{Marco conceptual} y el ensayo práctico en el \textit{Desarrollo del prototipo}. El desarrollo del prototipo estudia los materiales y componentes necesarios, el proceso de fabricación y el funcionamiento. 


