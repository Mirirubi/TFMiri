\chapter{Mi experiencia en la empresa}

	\section{Actividades}
	
		Durante la experiencia en la empresa he participado en diferentes actividades, partiendo de aquellas que únicamente requieren de observación y evolucionando hacia aquellas en las que he tomado parte activamente.
	
		\subsection{Reuniones}
		
				
			Existen dos tipos de reuniones en Movalsys, las semanales y las concertadas con clientes. Todas ellas son muy diferentes puesto que los objetivos son totalmente distintos. En las primeras de ellas, el objetivo fundamental es la actualización de los proyectos y las segundas pretenden buscar mercado y negocio.
			 
			En cuanto a las reuniones semanales realizadas en la empresa, he participado en todas ellas, ya que en ocasiones éstas coincidían con alguna actividad, visita o reunión en otra parte. Las reuniones se basaban en una mera puesta al día de los proyectos en los que la empresa está trabajando.
			
			Todas estas reuniones han seguido el mismo patrón aproximadamente. Mariano, el gerente, expone los puntos que tiene para comentar. Explica las novedades de cada uno de ellos, y los miembros del resto del equipo que intervienen en el proyecto en cuestión participan.
			
			Debido a que se trata de una empresa de reciente creación, las reuniones se dividen en dos partes fundamentales, que son:
			\begin{itemize}
				\item Estrategias empresariales.
				\item Datos técnicos de los proyectos.
			\end{itemize}
			
			Dependiendo de evento más próximo que tenga la empresa, es decir, según si el siguiente paso es el de realizar alguna presentación en alguna empresa o casos similares, las reuniones toman mayor aspecto técnico o no.
			
			Evaluando ahora las reuniones con posibles clientes, aunque en ellas he intervenido la mayoría de las ocasiones de forma pasiva, he observado notables diferencias con respecto a las reuniones internas de la empresa. En ellas se observa claramente la estrategia de negocio que se ha decidido en las reuniones y se ven las distintas posiciones de ambas partes. Después de éstas reuniones, se realiza una valoración entre los miembros asistentes a la misma en las que se proponen diferentes estrategias para poder conseguir el cliente, o colaborador, etc. Estas son las reuniones en las que quizá se aprende más sobre el mundo empresarial o quizás las más nuevas para mí.
			
			A veces, es cierto, que surgen reuniones esporádicas que suelen resolverse a la hora del descanso de la mañana. Suelen tener como objetivo comentar algún aspecto técnico puntual, o bien la propuesta de alguna reunión o el cuadrar horarios con el compañero con el que se está desarrollando alguno de los proyectos,etc. Éstas resultan más informales y no es necesario reunir a todos los miembros del equipo para ello.
			
		\subsection{Tareas}
					
			A lo largo del período de prácticas en la empresa las actividades han ido variando desde aquellas más sencillas a aquellas con mayor carga de responsabilidad. Puede dividirse en tres sub-períodos:
			
				\begin{enumerate}
					\item \textbf{Mediados de Febrero - Mediados de Marzo}.
					
					 Se trata de una etapa de observación y de aprendizaje de los procesos de medida mediante la asistencia a los mismos. Por ejemplo: medida realizada en el parque de bomberos de Cordovilla.
					
					También se realiza una visita a Adacen (Asociación de daño cerebral de Navarra) con el fin de asesorar técnicamente a los fisioterapeutas sobre un producto de rehabilitación motora que vino a ofrecer la empresa Vitia de San Sebastián. 
					
					\item \textbf{Mediados de Marzo - Mediados de Abril}.
					
					En esta ocasión se realiza un reparto de tareas en la que se me encarga la búsqueda de un sistema de sujeción de unos sensores para la práctica deportiva. El objetivo de la misma era tenerlo para la reunión semanal de la semana siguiente donde debía exponer mis progresos. Esto sirvió para empezar a sentirme parte de la empresa y empezar a tomar responsabilidades.
					
					Además, en este período asisto a unas jornadas en CEIN en las cuales varias empresas, entre ellas Movalsys, realizan propuestas a la empresa KyB. Movalsys realiza la propuesta en relación a prevención de riesgos laborales. Acabadas las exposiciones, nos reunimos con el directivo encargado de esa sección y se consigue llegar a un acuerdo para realizar un estudio en su empresa.
					
					Por último, se me encarga también la redacción de la parte de memoria técnica de una solicitud de proyecto de I+D+I del Gobierno de Navarra. En ella debía incluir la parte técnica relacionada con el TFM así como integrar otros sistemas ya que se trataba de un proyecto de transferencia de conocimiento. En este caso, el aprendizaje se centró más en temas burocráticos que no conocía hasta el momento.
						
					\item \textbf{Mediados de Abril - Mediados de Mayo}.
					
					Aproximadamente durante un mes que empezó en Abril, Movalsys ha estado realizando medidas en la Volkswagen. Una vez realizadas las medidas había que analizar las señales recogidas. En este caso, se me propone el  participar en este proyecto. A partir de este momento, las actividades que realizo empiezan a tomar relevancia ya que es el  trabajo que también realiza el resto del equipo.
					
					Además de la participación en este proyecto ya iniciado, realizamos una visita a la fábrica de KyB en los Arcos para observar los distintos puestos de trabajo y poder realizar la propuesta de ergonomía a la empresa para poder mejorar el aspecto de las lesiones de los trabajadores. Se pretende realizar un estudio parecido al realizado en Volkswagen.
					
					\item \textbf{Mediados de Mayo - Fin de la práctica}.
					
					En este último mes se ha centrado más en la redacción de memorias y en realizar medidas relacionadas con el TFM. Esto se ha compaginado con la elaboración de un informe general que se quiere diseñar en la empresa para la presentación de datos a los clientes y con la preparación de unos folletos para un encuentro de empresas en Bidart (Francia). También he participado en las decisiones.
					
				\end{enumerate}		
		
Además de todo ello, he participado en la corrección de propuestas, o consulta de ciertos temas de la empresa que normalmente se realizan por correo electrónico.
			
	\section{Problemas y soluciones}
	
Los principales problemas en los que se ha visto inmerso la empresa es en la búsqueda de financiación debido a que es una empresa de reciente creación. En este caso se busco una consultora con la que actualmente trabajan para la ayuda de Instrumento PyME. Se valoraron varias consultoras hasta encontrar la que más se adaptaba a las condiciones de Movalsys.

Otro de los problemas cotidianos de la empresa es el saber cuándo rechazar o aceptar ofertas ya que por una parte, tienen la necesidad de aceptar proyecto pero por otra parte, deben ofrecer algún tipo de beneficio tanto económico como para dar a conocer la empresa. Durante mi estancia en las prácticas surgió el caso de realizar un proyecto de colaboración con un centro tecnológico en Navarra y se planteó el poder llevarlo a cabo. Cuando realizaron la propuesta, cada uno de nosotros la recibió por e-mail con el fin de poder estudiarla. En una reunión semanal cada uno opinó y se rechazó la propuesta debido a que la cantidad de dinero que se pedía en un principio excedía lo que la empresa estaba dispuesta a poner. Por ello, se busco otra alternativa, que fue la de la consultora anteriormente mencionada.

Todo este tipo de contratiempos/decisiones se toman entre todos los miembros. Es la ventaja de tratarse de una empresa pequeña. Todas las opiniones tienen cabida y se pueden debatir.

	
	
	\section{Recursos materiales}

Los recursos materiales de los que dispone la empresa son adecuados a sus necesidades.
Gracias a que han ganado premios de emprendimiento disponen de sede en CEIN y debido a que es una empresa Spin-off de la UPNA dispone de ordenadores y un laboratorio donde también puede desarrollar su actividad. Además, ha conseguido un laboratorio en NavarraBioMed donde a partir de Junio va a llevar parte de su actividad.

Realmente no es una empresa que precise de muchos recursos materiales. Únicamente necesita los sensores para la medición y equipo informático donde analizar las señales y llevar a cabo el desarrollo del software.
	
	