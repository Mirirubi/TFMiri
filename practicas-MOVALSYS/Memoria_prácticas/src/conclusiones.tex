\chapter{Conclusiones \label{sec:conclusiones}}


En esta segunda experiencia de prácticas curriculares en empresa me llevo un aprendizaje más amplio que en la primera de las ocasiones. Quizá sea debido a que ya he tenido una experiencia previa en empresa por lo que la integración ha sido más sencilla o quizás sea debido a que el máster me ha ampliado la base técnica de la que disponía en el Grado. 

El equipo ha contado en todo momento con mi presencia, como si fuese una empleada más y además entendiendo en todo momento que estaba aprendiendo lo cual ha hecho que poco a poco ganase confianza y fuese adquiriendo responsabilidad.

Cuando comenzaron las prácticas, los primeros días fueron de nervios debido a las inseguridades por no entender en un principio los aspectos técnicos de la empresa y donde por tanto, no podía contribuir en gran medida. A base de preguntar e ir entendiendo el funcionamiento de la empresa, cada vez resultaba más cómodo incluso aportando opiniones y participando en soluciones técnicas.

Lo más destacable es que se me ha propuesto la continuación en la empresa continuando con el desarrollo de mi Trabajo Fin de Máster. El contrato es a través de un proyecto de I+D+I del Gobierno de Navarra en colaboración con Adacen (Asociación de Daño Cerebral de Navarra). Se trata de una buena oportunidad con la que empezar la inserción en el mercado laboral ya que el proyecto es atractivo y donde podré ganar más experiencia y continuar aprendiendo los aspectos técnicos que todavía no he adquirido en la empresa.

Realizando un balance global de las prácticas, puedo concluir que he aprendido tanto aspectos técnicos como personales(interactuar con diferentes personas, clientes,etc.) lo cual va  ayudar a que en un futuro pueda integrarme de forma fácil en el entorno empresarial. Además, la temática de la neuro-rehabilitación que va a ser donde desempeñe mi primera experiencia laboral ajena a las prácticas es un campo que desconocía y que esta empresa me ha ayudado a descubrir.