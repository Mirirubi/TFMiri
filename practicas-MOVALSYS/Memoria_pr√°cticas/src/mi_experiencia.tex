\chapter{Mi experiencia en la empresa}

Este capítulo tiene un carácter más personal, puesto que está dedicado a exponer mi experiencia durante estos meses en Movalsys S.L. 

	\section{Actividades}
	
		Durante las prácticas he participado en diferentes actividades de la empresa. Clasificables en función de su periodicidad en actividades constantes o esporádicas.
		
		\subsection{Actividades constantes}
		
		La definición de actividad constante se refiere a actividades continuadas, las que se realizan con cierta frecuencia, diaria o semanal. Dentro de estas actividades se encuentra el desarrollo del proyecto, las reuniones de empresa y los seminarios.
		
		\begin{itemize}
			\item \textbf{Desarrollo diario del proyecto}
			 
			\smallskip	
			Durante las prácticas mi ocupación principal ha sido el desarrollo de un guante para la captación del movimiento de las manos. Este proceso de desarrollo ha tenido varias fases, pasando por un cambio de tecnología a emplear. Ha sido un camino lleno de aprendizaje que divido en varias fases para simplificar su exposición:

		
			\begin{enumerate}
				\item \textbf{\textit{Conocer la empresa y el proyecto}}
				
				\smallskip	
				Cuando llegué a Movalsys S.L. era la primera vez que realizaba unas prácticas en empresa. Eran muchas cosas nuevas. De todas formas no me supuso ningún problema, ya que el equipo siempre ha sido muy bueno y han hecho todo lo posible por que todos estemos a gusto y encontremos nuestro lugar. En cuanto al proyecto, ya estaba comenzado su desarrollo así que lo primero que tuve que hacer fue documentarme sobre este. 
				\smallskip
				
				\item \textbf{\textit{Desarrollo del guante con sensores FBG}}
				
				\smallskip	
				El desarrollo de la primera versión del guante se vio más dilatado en el tiempo de lo esperado. Mejoré la disposición del cableado del prototipo, pero el resto de manufacturas necesarias para la obtención del prototipo sufrieron varios percances, debidos sobretodo a la delicadeza de la tecnología utilizada y los medios disponibles. Hasta poder conseguir finalmente el prototipo completado. Mientras tanto trabajé en el desarrollo del software. 
				
				\smallskip
				
				\item \textbf{\textit{Advertencia de la inoportunidad de la tecnología empleada}}
				
				\smallskip	
				Una vez finalizado el software. Conseguimos terminar el guante de sensores de fibra de bragg. Una de los inconvenientes de como hubo que 
				realizar el prototipo físico del guante es que en todo el proceso no había ningún punto de control, más que el final, dónde poder saber si los sensores seguían en buen estado. Al conseguir finalizar la fabricación se procedió a probar el software y se pudo observar, cómo después de todo el desarrollo del guante este estaba deteriorado. No tenía sentido lógico seguir intentado realizar el guante con los medios disponibles, así que se reorientó el proyecto a través de otra tecnología. 
				
				El aborto de la primera versión del guante podría haberse llevado a cabo mucho antes. Pero al tratarse de una spin-off, que valora el proceso de investigación no fue así. De todas formas el seguir intentándolo, en cierto modo, me sirvió para lograr un mayor aprendizaje. También es cierto que hasta que no tuvimos todo el sistema completamente montado no fuimos totalmente conscientes de que no funcionaría. 
				\smallskip
				
				\item \textbf{\textit{Reorientación del proyecto - Guante con sensores inerciales}}
				
				\smallskip	
				Durante el final de la fase anterior ya estaba trabajando en la reorientación del proyecto. La tecnología a emplear esta vez fueron los sensores inerciales. 
				Como en este caso yo soy la primera persona en trabajar en este desarrollo he tenido que realizar una labor de investigación sobre la tecnología y los trabajos previos con esta en otras universidades. 
				Después de la documentación he realizado una propuesta de como llevar a cabo esta versión del producto, siendo esta validada por mi equipo.
				En este punto sucede la incorporación de Martín al equipo. Él será quien se encargue del diseño de la estructura del guante, la parte relativa a materiales y ergonomía. Una parte crucial que de hecho fallaba en el guante de sesores FBG. 
				\smallskip
								
			\end{enumerate}
			
			
			
			\medskip
			
			\item \textbf{Reuniones semanales de empresa} 
			
			\smallskip
			La empresa realiza reuniones semanales para mantener a todo el equipo al día de las cosas que afectan a la empresa. He participado en todas ellas. 
			
			En las reuniones se suele llevar un orden para tratar los temas por puntos. Durante estas se exponen los avances semanales de cada trabajador al resto del equipo. El ritmo lo suele marcar Mariano. 
			
			Otras prácticas que se han llevado a cabo durante las reuniones son las de consulta popular. Un ejemplo de ello son las consultas al resto del equipo sobre la aceptación de nuevos proyectos entrantes, que sirve para conocer si el equipo tiene los suficientes recursos libres como para poder abordar los mismos. 
			
			Puesto que el trabajo que cada uno realiza en ocasiones es bastante individual también se aprovechan las reuniones para recibir un feedback del resto de los compañeros sobre los desarrollos en los que uno está inmerso. Por ejemplo, si se está realizando una presentación para exponer ante un cliente, también se puede aprovechar las reuniones para revisarla. Esto resulta en una mejora importante en la calidad de la presentación realizada al cliente(en el caso del ejemplo expuesto).
					
			Igualmente, si hace falta, es posible ponerse en contacto con los integrantes del equipo fuera de las reuniones. Incluso, a veces se ha dado la situación de aprovechar después de una reunión para reunirnos dos o tres personas para resolver alguna duda más concreta sin ocupar al resto del equipo en asuntos sin relevancia para ellos en ese momento.
			
			En algunas ocasiones puntuales surgen reuniones esporádicas con asi todo el equipo que se resuelven en la hora del descanso matutino para tratar algún aspecto puntual o urgente.
			
			\medskip
			
			\item \textbf{Seminarios cada tres semanas}
			
			\smallskip
			Los seminarios, que se celebran cada tres semanas aproximadamente, son una oportunidad para conocer en mayor profundidad el trabajo que realiza cada integrante del equipo. Suelen durar en torno a una hora. En ellas los trabajadores tienen oportunidad de exponer con más detalle en lo que están trabajando o compartir alguna experiencia.
			
			A continuación los seminarios llevados a cabo durante estos meses:
			
			\begin{itemize}
				\item \textbf{Introducción a redes neuronales} - \textit{Patxi Iriarte} 
				
				\smallskip	Patxi  expone el trabajo que está realizando durante sus prácticas desde un punto de vista teórico. Su cometido es identificar empleando redes neuronales cuando se está midiendo un paso con el sensor. Realizando las redes neuronales una tarea de identificación.
				\smallskip
				
				\item \textbf{Funcionamiento de Jero} - \textit{Mariano Velasco}
				
				\smallskip
				Mariano explica el funcionamiento del nuevo producto desarrollado en la empresa: Jero. Este producto consiste en una herramienta para profesionales que trabajan con personas mayores y que necesitan un método rápido y sencillo para evaluar capacidad funcional y cognitiva de estos.
				\smallskip
				
				\item \textbf{Análisis de la carrera} - \textit{Pablo lecumberri} 
				
				\smallskip
				Pablo expone como ha resuelto la identificación de pasos en análisis de carrera, dentro de la línea de la empresa que se dedica a la medición de deportistas. Para ello ha llevado a utilizado la estadística como método de anticipación, para luego identificar los pasos.			
				\smallskip
				
				\item \textbf{Desarrollo de diseño de producto} - \textit{Martín Leturia} 
				
				\smallskip
				Martín nos cuenta su planning de trabajo en el desarrollo de producto y los avances que lleva hasta el día del seminario. 
				\smallskip
				
				\item \textbf{Modelo de negocio de REHHAND} - \textit{Miriam Rubio} 
				
				\smallskip
				Expongo el modelo de negocio que he realizado como ejercicio para el programa Explorer organizado en la Universidad Pública de Navarra. 
				\smallskip

			\end{itemize}
			
			\medskip	
		\end{itemize}
		
	
		
		
		\subsection{Actividades esporádicas}
		
		La definición de actividad esporádica se refiere a actividades que se han realizado de forma ocasional. Dentro de estas actividades se encuentran las visitas al cliente, asistencia a eventos y otros.
		
		\begin{itemize}
			\item \textbf{Visitas al cliente y colaboradores} 
			
			\smallskip
				Por el tipo de organización y actividad de la empresa, las visitas a los clientes y colaboradores son cruciales para comprender la situación de estos y cómo resolver con nuestros proyectos los problemas que tienen. Son varias las visitas a clientes y colaboradores realizadas con la empresa:
				
				\begin{itemize}
					\item \textbf{ADACEN} - \textit{Primer contacto} 
					
					\smallskip	
					En esta primera visita conozco la asociación de daño cerebral (ADACEN). Miembros del equipo se habían desplazado hasta allí para tomar medidas de los pacientes. Así conozco a los fisioterapeutas con los que voy a trabajar posteriormente. Observo en directo a los pacientes y las dificultades de movimiento que padecen. Y además aprendo cómo se emplea la tecnología que genera la empresa.
					\smallskip
					
					\item \textbf{ADACEN} - \textit{Segundo contacto} 
					
					\smallskip	
					En esta ocasión la visita se centra en la observación más detallada de la relevancia de la medición de los pacientes para mi proyecto. Además de una pequeña exposición de la tecnología que disponen en el centro.
					\smallskip
					
					\item \textbf{ADACEN} - \textit{Testeo de la tecnología en desarrollo} 
					
					\smallskip	
					Cuando el proyecto se encuentra avanzado se concierta una reunión en ADACEN para mostrar los avances. En esta reunión se determinan en que sentido se realizarán las siguientes decisiones de diseño. Se comprenden mejor las necesidades de los fisioterapeutas.
					\smallskip
					
					\item \textbf{ADACEN} - \textit{Primer contacto de Martín} 
					
					\smallskip	
					Puesto que la incorporación de Martín sucede tras descartar la tecnología empleada en la primera fase del proyecto, cuando se tiene una nueva propuesta. Esta visita sirve para presenciar las diferentes condiciones de las manos de los pacientes y los procesos de rehabilitación que realizan con ellos. Una visita muy ilustrativa e importante para la continuidad del proyecto.
					\smallskip
					
					\item \textbf{Asociación Ibili} - \textit{Primer contacto} 
					
					\smallskip	
					En esta ocasión visitamos la asociación Ibili, dónde conozco a los trabajadores de allí y el proyecto en el que están inmersos. Entre otras actividades, en la asociación han construido un acoplo para ponerle a las sillas de ruedas que permite mover la silla de forma motorizada.
					\smallskip
					
				\end{itemize}
			\medskip

			\item \textbf{Eventos} 
			
			\smallskip
				Durante mi estancia en las prácticas en Movalsys S.L. tuvo lugar el aniversario 25 de ADACEN. Con motivo de este la asociación realizó una jornada en el Civican. Asistimos a este evento, que fue muy interesante ya que me permitió conocer más sobre otros aspectos del daño cerebral que no conocía antes. Además de otros proyectos de otras empresas que ayudan a mejorar la calidad de vida de los afectados por esta dolencia en otros ámbitos.
								
			\medskip
			
			\item \textbf{Otros} 
			
			\smallskip
			Debido a que la empresa es una spin-off de la Universidad Pública de Navarra, en varias ocasiones participamos en algún que otro vídeo de esta. Cómo por ejemplo el vídeo que realizó la asociación universitaria i$^{2}$tec con motivo del día de la mujer, para dar un poco de visualización a las mujeres dentro de la comunidad STEAM. 
			\medskip
		\end{itemize}
		
	
		
			
	
			
	\section{Toma de decisiones}
	
	
	En cuanto a la toma de decisiones, como he expuesto con anterioridad, se realizan tomando en cuenta las aportaciones al respecto de todos los miembros de la empresa. Es un equipo con capacidad de debate, abierto a escuchar las propuestas y opiniones de todos.

	
	
	\section{Recursos materiales}

	En general los recursos materiales de los que dispone la empresa son adecuados a sus necesidades. Debido a que es una empresa Spin-off de la UPNA dispone de ordenadores y un laboratorio donde también puede desarrollar su actividad. 
	
	Realmente no es una empresa que precise de muchos recursos materiales. Únicamente necesita los sensores para la medición y equipo informático donde analizar las señales y llevar a cabo el desarrollo del software.
	
	En el caso del primer guante de sensores de FBG se disponía de los materiales. El problema fue la falta de medios se dio a raíz de que para algunas tareas concretas la empresa cuenta con las instalaciones de la universidad y no estaban preparadas para las características del prototipo.
	