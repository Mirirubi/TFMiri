\chapter{Valoración global de las prácticas \label{sec:conclusiones}}

\medskip

Mi valoración general de las prácticas es muy buena. Ha sido mi primera experiencia en una empresa y ha sido muy satisfactoria. Además de dar con un equipo genial he aprendido mucho en el ámbito empresarial al tratarse de una empresa pequeña.

\medskip

Mis compañeros han contado en todo momento conmigo, como una empleada más, pero conscientes de mi falta de experiencia. Esto ha ayudado a que ganase confianza y fuera adquiriendo más responsabilidad. Me he sentido valorada cada vez que he realizado alguna propuesta y se tenido en consideración o incluso llevado a cabo.

\medskip

Todas las actividades me han resultado enriquecedoras:
 
 \medskip
 
 En cuanto al desarrollo del proyecto me ha servido mucho para el aprendizaje, tanto tecnológicamente como personalmente. He utilizado diversas tecnologías y he aprovechado para utilizar otras que aunque no fueran cruciales el aprender a usarlas puede servirme en un futuro. 
 
 
 \medskip
 
 El hecho de participar en las reuniones tuvieron un papel importante en mi sentimiento de pertenencia en la empresa. Si bien es cierto mis aportes en las reuniones han evolucionado según evolucionaba mi experiencia en estas. 
 
 \medskip
 
 Los seminarios han sido una herramienta muy útil para conocer más sobre la dedicación del resto de compañeros y ampliar los propios conocimientos.
 
 \medskip
 
 Las visitas a clientes y colaboradores para mi han sido unas actividades que me han hecho valorar mucho más la repercusión de la tecnología en la sociedad. Me han servido además como motivación en mi trabajo. Me parece una práctica imprescindible para el desarrollo de un producto, y al ser una empresa pequeña es una suerte poder ser el desarrollador quién las realice.
 
 \medskip
 
Realizando un balance global de las prácticas, puedo concluir que he aprendido tanto aspectos técnicos como personales que me van a ser de mucha utilidad y estoy muy agradecida por ello.